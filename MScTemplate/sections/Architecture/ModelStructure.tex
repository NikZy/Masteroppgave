\section{Data correlation}
\label{section:Architecture:ModelStructure}
\todo[inline]{Fjerne denne delen siden vi snakker om det i discussion:model structure?}

With the aim to increase the predictive accuracy of the proposed model,
different approaches to data selection and groupings could be needed.
The available data contains several time series with varying degrees of correlation between them.
This correlation can be evaluated through means of domain knowledge,
knowing the data should correlate in some way, or through statistical methods and analysis.
Correlation between the time series could provide additional information used in prediction,
increasing the predictive accuracy for the framework.


To take advantage of the correlation between data,
the predictive model needs to take this into account.
%One way to solve this is through the creation of a global prediction model,
%training a model on several data series.
With this, it would be able to evaluate if the proposed framework has an increase in predictive accuracy
when evaluating data globally, in groups of time-series correlation, using local methods as a baseline.
In addition, correlation data should also be evaluated through the use of a multivariate method,
where several correlated time series are evaluated simultaneously.

\todo[inline]{Add image of local VS correlation VS Global VS Multivariate?}

Predictions done on correlating time series could result in better accuracy,
but what approach is the best suited for this problem is unknown.
Comparing the different approaches should give insight into the predictive ability of the proposed framework,
as well as the inherent correlation between the data, if there exists one.
All the approaches should therefore be considered.

