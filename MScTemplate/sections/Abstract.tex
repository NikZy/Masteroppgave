\section*{Abstract}
\label{section:Abstract}

The E-commerce sector is heavily influenced by trends in user activity and purchase patterns.
'Prisguiden' is an online service created to easily compare product prices amongst retailers.
With this, the site is able to record the amount of user activity connected to a product or product category.
In able to improve their service, knowing product trends could help with resource allocation for what products and sites are listed in their system.
In order to predict such trends before they occur, a time-series prediction system could be utilized.

Time-series prediction in E-commerce sales has been attempted several times,
often solved by statistical models or deep learning methods.
Multiple approaches to time-series prediction are proposed in the related literature,
resulting in many different models and frameworks.

This study conducts an extensive structured literature review of the current state of times series prediction research, E-commerce forecasting,
and current state-of-the-art methods.
Subsequently, a hybrid deep neural network model is proposed as a means to increase the predictive accuracy of E-commerce time-series forecasting.
Combining a convolutional autoencoder with an LSTM network, we discuss the predictive ability of the proposed model
and its architecture.
Combining this model with data preprocessing and grouping, we propose the creation of a global prediction framework,
grouping related time-series together.




\iffalse
  This paper provides a template for writing AI project rapports for either the AI specialization project; masters "datateknikk" or masters "informatikk". The use of the template is recommended and is written in english as we encourage students to submit their project and masters theses in English.
  The template does not form a compulsory style that you are obliged to use. However, the format and contents are a result of a joint AI group initiative thus providing a common starting point for all AI students. For a given project tuning of the template may still be required. Such tuning might involve moving a chapter to a section or vice versa due to the nature of the project.

  The abstract is your sales pitch which encourages people to read your work but unlike sales it should be realistic with respect to the contributions of the work. It should include:
  \begin{itemize}
    \item the field of research
    \item a brief motivation for the work
    \item what the research topic is and
    \item the research approach(es) applied.
    \item contributions
  \end{itemize}

  The abstract length should be roughly half a page of text --- without lists, tables or figures.
\fi
