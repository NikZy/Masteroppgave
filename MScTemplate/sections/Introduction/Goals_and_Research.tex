\section{Goals and Research Questions}
\label{sec:Goals and Research Questions}


The aim of this thesis (paper?) is to explore the ability of using a machine learning algorithm in order to accuratly forecast the user interest in product categories on a price comparison website.
Exploiting the predictive abilities of deep neural networks such as Convolutional neural network, Long-Short term memory, and autoencoders,
we intend to introduce a new predictive model into the problem space.
By doing this, we intend to outperform the current state of the art predictive algoritms in the problem space.

% TODO: Should we disclose that we are working with Prisguiden in this paper?
% Still unsure what conclusion we arrived at there.
The data supplied by "Prisguiden" contains historical data about user activity such as visitation data and click data.
We intend to use this data in order to predict future product/category trends, based on the historic data.

\begin{description}
    \item[Goal]{\it To accuratly predict future product category trends based on historical visitation and click data using a Convolutional Autoencoder with LSTM.}
\end{description}

In order to measure the viability and accuracy of the proposed goal, we need to look into the already propsed methods in this problem space.
This is required in order to assess weather or not our not our goal is rendered mute due to previous solutions.


\begin{description}
    \item[RQ1]{\it What are the existing solutions for predicting future product category trends, or sales trends, based on historical time series data?}
    % What are the existing solutions in problem space P? 
\end{description}

\begin{description}
    \item[RQ2]{\it How does the different solutions found by addressing RQ1 compare to each other with respect to C?} 
\end{description}

These two research questions is intended to be answered in this paper, addressed through a academic literature reviw.
\newline

Later, out focus will shift towards assessing the predictive capabilities of the method proposed in the goal.

\begin{description}
    \item[RQ3]{\it How long ahead can we give a meaningful prediction?}
\end{description}

\begin{description}
    \item[RQ4]{\it How well does our model predict anomalies?}
\end{description}



\iffalse
A masters is a research project and thus there needs to be a question(s) that need answered.
Such questions are often a very important part of the results that come out of the specialisation project.
For those following the one year masters project, it is desirable to create such questions as early as possible as   The formation of such questions provide both an important driving force for the masters project and provide clarity as to the goals sought. However, one will expect to refine the questions and thus the final path of the masters as work progresses. However any refinements should be conducted with care so as to avoid that the original aims, and previous work are not lost.  
It is always good to have one (or max 2) key questions and perhaps some sub questions. 

\begin{description}
\item[Goal] {\it Lorem ipsum dolor sit amet, consectetur adipiscing elit.}
\end{description}

Your goal/objective should be described in a single sentence. In the text under you can expand on this sentence to clarify what is meant by the short goal description. 
The goal of your work is what you are trying to achieve. This can either be the goal of your actual project or can be a broader goal that you have taken steps towards achieving. Such steps should be expressed in the research questions. 
Note that the goal is seldom to build a system. A system is built to to enable experiments to be conducted. The research question/goal would be the goal that the system is implemented to meet.  


\begin{description}
\item[Research question 1] {\it Lorem ipsum dolor sit amet, consectetur adipiscing elit.}
\end{description}

Each research question provides a sub-goal and these should be precise and clearly stated enabling the reader to match your results to the original goals. They will also form the driving force for the experimental plan. 

\begin{description}
\item[Research question 2] {\it Lorem ipsum dolor sit amet, consectetur adipiscing elit.}
\end{description}

{\it Lorem ipsum dolor sit amet, consectetur adipiscing elit. Nam consequat pulvinar hendrerit. Praesent sit amet elementum ipsum. Praesent id suscipit est. Maecenas gravida pretium magna non }
\fi
