
\section{Problem description} \label{into:problem-description}
Online shopping platforms retain large amounts of data regarding product sales and interest data.
Despite this, it is not always easy to know what information this data might contain, and what it could mean to exploit and analyze this data.
Analyzing data such as product interest or sales could help retailers discover useful information such as product trends or anomalies.
While methods for attaining such information already exists to some degree, in the form of simple statistical methods or neural networks,
there is still room for improvements.

%%%
Unlike most previous solutions, the aforementioned problem space contains several time series,
each for a different product, where the relative ...

%%%

This theses will attempt to come up with a solution in order to predict product trends using time series data of several product lines in groupings.
By doing this, we hope to increase the predictive accuracy of the model by using grouped time series with different corrolations,
insted of only one series of data to which the statistical methods are limited.

More spesifically, we intend to look at the use of a modert time series predictive method, a \textit{Convolutional autoencoder with LSTM} in order to make predictions.
The main goal of the thesis is to improve the predictive forecasting abilities on e-commerce data in a time series using a modern deep learning approach.