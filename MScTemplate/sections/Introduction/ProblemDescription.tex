
\section{Problem description}
\label{section:Introduction:ProblemDescription}
Online shopping platforms retain large amounts of product sales and interest data.
Despite this, it is not always easy to know what information this data might contain and what it could mean to exploit and analyze this data.
Analyzing product interest data or sales data could help retailers discover helpful information such as product trends or anomalies.
While methods for attaining such information already exists to some degree, in the form of simple statistical methods or neural networks,
there is still room for improvements.


% Er det heelt fastsatt at vi skal se på grupperinger av tidsserier enda?
% Er ikke dette noe vi må teste litt først?
% F.eks vil ikke baselinen vår ARIMA være gruppert
This thesis will focus on proposing a solution for predicting product trends using time-series data of several product categories.
% By doing this, the intent is to increase the predictive accuracy of the model by using grouped time-series with varying correlations,
% instead of only one series of data to which the statistical methods are limited.


% -> Ja, det er dette vi skal gjøre men det kommer for tidlig!
% More specifically, we intend to evaluate the use of a modern time-series predictive method, a \textit{Convolutional autoencoder with LSTM} \textbf{(CA-LSTM)} in order to make predictions.
% The main goal of the thesis is to improve the predictive forecasting abilities on e-commerce data in a time-series using a modern deep learning approach.
