\section{Forecasting E-commerce}
\label{section:RelatedWork:forecasting-ecommerce}

User click rate on consumer e-commerce goods is a narrow domain,
and we are unable to find any research that has yet been done within this particular field.
However, one can argue that user interest in consumer goods follows the same distribution
as online retail sales.
Both domains will have the same seasonality with weekly and yearly periodic fluctuations.
Additionally, the domains contain the same kind of products and categories.
Both domains are dictated by the same external factors, such as media, commercials, 
and trends.
Forecasting retail sales should translate well to e-commerce click rate.


\cite{Ramos2015} did a comparative study on exponential smoothing methods
based on state-space models such as ETS and ARIMA.
The aim was to measure the forecasting performance of the two modeling frameworks
when applied to retail sales data of five different categories,
containing univariate time series regarding sales of womans footwear.
The domain proved difficult due to strong trends and seasonal patterns.

% What are their results?
The models forecasting ability were evaluated through the loss functions RMSE, MAE, and MAPE.
Both ETS and ARIMA achieved quite similar results with both one-step and multi-step forecasting.
Multi-step forecasts are generally more accurate.
This is because it is able to incorporate information that is more up-to-date.
When an automatic algorithm is applied to the overall out-of-sample forecasting,
the ARIMA models do not achieve better results than the ETS models.
Neither model is best suited for all circumstances.
Each time series would need to identify an appropriate ARIMA model.
With differing data, identifying models would require data analysis of all the different time series.
% - No clue what this attempts to say, so if it does not contribute enough, it is removed
%Transformations for variance stabilization, as well as making unit-root tests, would be needed to select the degrees of differnecing in order to achieve stationarity.
\todo[inline]{A sentence is commented out. Evaluate if it should be added.}


%% What are they doing?
%Comparing ARIMA and ETS to eachother on a univariate retail time series.
%% Why are they doing it?
%They aim to test two ETS and ARIMA on the consumer retail forecasting domain. 
%Which is a difficult domain due to strong trend and seasonal patterns.
%Comapre the forecasting perfomance of ETS and ARIMA models when
%applied to a case study of retail sales of five different categories of woman footwar
%from the Portuguese retailer Foreva.


\cite{Chu2003} showed that Holt-Winters exponential smoothing and ARIMA perform well when macro-economic
conditions are relatively stable. However, when economic conditions are volatile
ANNs outperform linear methods.
Multi-step regression is preferred.

% Skrive om. Finne kilde
%\todo[]{Bør ligge under en seksjon med preprossesering?}
%[Zhang and Qi...] and [Kuvalmaz et al.]
%Investigated the use of ANNs in forecasting time series with a strong trend and seasonality.
%They concluded that the overall forecasting perfomance of ANNs, 
%evaluated via RMSE, MAE and MAPE is not better than ARIMA models in predicting
%retail sales without appropriate data preprocessing,
%namely detrending and deseasonalization.

% Skrive om. Finne kilde
%[Pan et al.]
%Tested a hybrid forecasting method of integrating EMD and NN and compares them with direct NN models.
%and vs the ARIMA model for aggregate retail sales forecasting.
%Different macroeconomic conditions were studied.
%Hybrid NN model is more stable compared to direct NN and arima during volatile economy.
%ARIMA performs consistently well during stable economic activity.

% This section should describe earlier work done within the field of E-commerce and Sales prediciton!
% This is the main explainer for why we need do any work, and why we cant just apply a model.
% What can be done to improve the models and take the most important pars from the different approaches.
% This will more or less be what i though motivation was going to become!
% Thus; Se Motivation_prev.tex for reference

\cite{Weng2020} writes that ARIMA, SARIMA and Holt-Winter's
Exponential Smoothing model etc. are inconsistent with the actual changes in the sales
of retail stores.
Therefore the results are usually unstable due to their in-applicability
in the processing of nonlinear relationships.
However, statistical methods have the advantage of interpretability.
\cite{Bowen2020} wanted to use machine learning to maintain
the trade-off between interpretability and consistency with actual
changes in retail sales.
Two new machine learning models were applied in order to solve this.
\todo[]{Need reference to XGB model}
LightGBM, an ensemble learning method, which is a further optimization of the XGB model.
And one model called Prophet \citep{Zunic2020}.

% What results did they get?
The study finds that traditional models such as ARIMA have trouble forecasting
due to the fast-moving domain of consumer retail buying patterns.
Additionally, their connection to external factors,
such as holidays and price changes, increases the complexity further.
The machine learning models performed better, with lower predictive error
and good interpretability.
However, the LightGDM model outperformed the Prophet model because of the prophet
models limitation to univariate time series.


% What do they do?
%Using the machine learning model lightGBM to predict Walmart supermarket sales
%from 2011 to 2016. Using Root Mean Square Error (RMSE) as metric.
%Prophet model [8]
%LightGBM which is a further optimization of the XGB model.
%
%
%% Why are they doing it?
%The objective is to develop an effective sales forecasting model maintaining the 
%trade-off between interpretation and consistency with actual changes of retail sales.

The best example of related work we could find on our domain was 
\cite{Bandara2019}.
% What did they do?
They generate an accurate E-commerce sales forecasting model using LSTM  Neural Network.
They pool similar time series into clusters and train univariate, global models on each cluster.
% Why did they do it?
The aim was to improve E-commerce sales forecasting by exploiting sales correlations and relationships
available in an E-commerce hierarchy.

Data preprocessing was done using a forward-filling strategy to impute missing sales
observations in the dataset.
The data was then normalized to accommodate the high sales volume ranges.
They then used the mean of the sales of a product as the scaling factor.

Two approaches to groupings were explored.
The first approach was based on domain knowledge.
Sales ranking and the percentage of zero sales were used as the primary business metrics
to form groups of products.
Group 1 represents their most popular products (67\% of sales), group 2 represents
their more unpopular products (33\% of sales), and group 3 represents the rest.

The second approach was based on time series clustering.
K-means clustering was used on a set of time-series features to identify groupings.
The first two features were business-specific features, namely "sales.quantile" and "zero.sales.percentage".
The rest of the features were time-series specific.
%They extracted features with the \textit{tsfeatures} package.
A silhouette analysis was utilized to determine the optimal number of clusters in the K-means algorithm.

 % What results did they get?
The results outperformed the state-of-the-art univariate forecasting techniques.
Thereby concluding that E-commerce product hierarchies contain various cross-product demand
patterns and correlations.
Exploiting these relationships are necessary to improve the sales forecasting
accuracy in this domain.

With the effort to advance the predictive capabilities of models in e-commerce,
previous work done within the problem space is highly relevant as a starting point.
This shows what has already been accomplished and what new methods have not yet been attempted.


\iffalse
% ---- Litt rotete og hører mer hjemme i diskusjons biten
The \cite{Bandara2019} differ from our situation in the sense that individual product interests are modeled.
The authors example of related time series are how
all subcategories under the genre fictional books will follow similar patterns.

We are interested in the category level, and will pool the interests of all products under
a category into one time series. It is not yet clear if we can assume that similar categories 
will follow similar patterns. 
For example computer screens and laptops are childs of the same category parent, 
but interests in these products will be generated from two different processes, as their 
use case are different.
Even though we can not use the same assumptions on our domain, their results are still relevant
for our work.
\fi