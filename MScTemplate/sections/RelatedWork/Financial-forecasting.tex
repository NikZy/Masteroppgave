\section{Forecasting E-commerce}
\label{section:RelatedWork:forecasting-ecommerce}
User click rate on consumer e-commerce goods is a really narrow domain,
and we are unable to find any research that has yet been done on this particular field.
However one can argue that user interest in consumer goods follows the same distribution
as online retail sales.
Both domains will have the same seasonality with weekly and yearly periodic fluctuations.
Additionally, the domains contain the same kind of products and categories.
Both domains are dictated by the same external factors, such as media, commercials, 
and trends.
Forecasting retail sales should translate well to e-commerce klicks.


\cite{Ramos2015} did a comparative study on exponential smoothing methods
based on state-space models such as ETS and ARIMA.
The aim was to measure the forecasting performance of the two modeling frameworks
when applied to retail sales data of five different categories,
containing univariate time series regarding sales of womans footware.
The domain proved difficult due to strong trends and seasonal patterns.

% What are their results?
The models forecasting ability were evaluated through the loss functions RMSE, MAE and MAPE.
Both ETS and ARIMA achieved quite similar results with both one-step and multi-step forecastign.
Multi-step forecasts are generally more acurate.
This is because it is able to incorporate information that is more up to date.
When an automatic algorithm is applied to the overall out-of-sample
forecasting the ARIMA models does not achieve better results than the ETS models.
Neither of the models is best for all circumstances.
Each timeseries would need identify a appropriate ARIMA model.
With differing data, identifying models would require data analysis of the different time series.
Transformations for variance stabilization, as well as making unit-root tests to select the decrees of differnecing in order to achieve stationarity.


%% What are they doing?
%Comparing ARIMA and ETS to eachother on a univariate retail time series.
%% Why are they doing it?
%They aim to test two ETS and ARIMA on the consumer retail forecasting domain. 
%Which is a difficult domain due to strong trend and seasonal patterns.
%Comapre the forecasting perfomance of ETS and ARIMA models when
%applied to a case study of retail sales of five different categories of woman footwar
%from the Portuguese retailer Foreva.


% Skrive om. Finne kilde
\todo[inline]{Rewrite! Do NOT cite directly from the text, always rewrite!}
\todo[]{Bør denne ligge under statistics vs ann seksjon?}
[Forecasting aggregate retail sales: a comparison of ANN and traditional methods. I. ALON]
Winters exponential smoothing and ARIMA perform well when macro-economic
conditions are relatively stable. When economic conditions are volatile
ANNs outperform linear methods and multi-step regression is preferred.

% Skrive om. Finne kilde
\todo[]{Bør ligge under en seksjon med preprossesering?}
[Zhang and Qi...] and [Kuvalmaz et al.]
Investigated the use of ANNs in forecasting time series with a strong trend and seasonality.
They concluded that the overall out-of-sample forecasting perfomance of ANNs, 
evaluated via RMSE, MAE and MAPE is not better than ARIMA models in predicting
retail sales without appropriate data preprocessing,
namely detrending and deseasonalization.

% Skrive om. Finne kilde
[Pan et al.]
Tested a hybrid forecasting method of integrating MED and NN and compares them with direct NN models.
and vs the ARIMA model for aggregate retail sales forecasting.
Different macroeconomic conditions were studied.
Hybrid NN model is more stable compared to direct NN and arima during volatile economy.
ARIMA performs consistently well during stable economic activity.

% This section should describe earlier work done within the field of E-commerce and Sales prediciton!
% This is the main explainer for why we need do any work, and why we cant just apply a model.
% What can be done to improve the models and take the most important pars from the different approaches.
% This will more or less be what i though motivation was going to become!
% Thus; Se Motivation_prev.tex for reference

\cite{Weng2020} writes that ARIMA, SARIMA and Holt Winter's
Exponential Smoothing model etc. are inconsistent with the actual changes in the sales
of retail stores and usually the results are unstable due to their in-applicability
in the processing of nonlinear relationships.
However statistical methods have the advantage of interpretability.
\cite{Bowen2020} wanted to use machine learning to maintain
the trade-off between interpretability and consistency with actual
changes of retail sales.
To solve this they tried two new machine learning models.
\todo[]{Need reference to XGB model}
One called LightGBM which is a further optimization of the XGB model.
And one model called Prophet \citep{Zunic2020}.

% What results did they get?
The study finds that traditional models such as ARIMA has trouble forecasting
because the fast moving domain of consumer retail buying patterns and 
their connection to external factors, such as holdays and price changes.
The machine learning models performed better, with smaller prediction errors
and good interpretability.
However the LightGDM model outperformed the Prophet model because the prophet
models limitation to univariate time series.

\todo[inline]{This whole section is lacking in drøfting og refleksjoner}

% What do they do?
%Using the machine learning model lightGBM to predict Walmart supermarket sales
%from 2011 to 2016. Using Root Mean Square Error (RMSE) as metric.
%Prophet model [8]
%LightGBM which is a further optimization of the XGB model.
%
%
%% Why are they doing it?
%The objective is to develop an effective sales forecasting model maintaining the 
%trade-off between interpretation and consistency with actual changes of retail sales.

The best example of related work we could find on our domain was 
\cite{Bandara2019}
% What did they do?
They generate an accurate E-commerce sales forecasting model using LSTM  Neural Network.
They pool similar time series into clusters and train a univariate, global model on each of these 
clusters.
% Why did they do it?
They wanted to improve E-commerce sales forecasting by 
exploiting sales correlations and relationships available in an E-commerce hierarchyeploit sales correlations and relationships available in an E-commerce hierarchy.

Of data preprocessing they used a forward-filling strategy to impute missing sales
observations in the dataset. They normalized the data to accomodate the high
sales volume ranges. They used the mean of the sales of a product as the scaling factor.

They tried to approaches for grouping. One based on domain knowledge.
They used the sales ranking and the percentage of zero sales as primary business metrics
to form groups of products. Group 1 represents their most popular products (67\% of sales), group 2 represents
their more unpopular products (33\% of sales) and group 3 represents the rest.

The second approach was based on time series clustering, where they performed K-means
clustering on a set of time series features to identify groupings. The first two features
were business specific features, namly sales.quantile and zero.sales.percentage.
The rest of the features were time series specific.
%They extracted features with the \textit{tsfeatures} package.
They also did a silhouette analysis to determine the optimal number of clusters in the K-means algorithm.

 % What results did they get?
Their results outperformed the state-of-the-art univariate forecasting techniques.
They conclude that E-commerce product hierarchies contani various cross-product demand
patterns and correlations. Exploiting these relationships are necessary to improve the sales forecasting
accuracy in this domain.

The \cite{Bandara2019} differ from our situation in the sense that they model
induvidual product interests. The authors example of related time series are how
all subcategories under the genre fictional books will follow similar patterns.

We are interested in the category level, and will pool the interests of all products under
a category into one time series. It is not yet clear if we can assume that similar categories 
will follow similar patterns. 
For example computer screens and laptops are childs of the same category parent, 
but interests in these products will be generated from two different processes, as their 
use case are different.
Even though we can not use the same assumptions on our domain, their results are still relevant
for our work.