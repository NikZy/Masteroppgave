\section{Forecasting e-commerce}
\label{section:RelatedWork:forecasting-ecommerce}
User click rate on consumer e-commerce goods is a really narrow domain, and no
research has yet been done on this particular field.
However one can argue that user interest in consumer goods follow the same distribution
online as retail sales.
Both domains will have the same weekly, and yearly periodic fluctuations.
Both domains contain the same kind of products.
And both domains are driven by the same external factors, such as media, commercials 
and trends.
Forecasting retail sales should translate nicely to e-commerce klicks.

\cite{Ramos2015} did a comparitive study on exponential smoothing 
methods bases on stace space models called ETS models, and ARIMA.
Their goal was to measure the forecasting perfomance of these two modeling frameworks
when they are applied to retail sales data
five different cateogries of univariate womans footware time series.
The domain was difficult because of the strong trend and seasonal patterns.
% What are their results
Their results was ETS and ARIMA evaluated via RMSE, MAE and MAPE is quite similar on both
one-step and multi-step forecasts. Multi-step forecasts are generally better
because multistep incorporate information that is more updated.
When an automatic algorithm is applied to the overall out-of-sample
forecasting perfomance of ARIMA models is not better than ETS models.
Neither is best for all circumstances.
For each retail series they had to identify an appropriate ARIMA model.
Deciding the required transformations for variance stabilization, and making unit-root
tests to select the decrees of differencing to achieve stationarity.

%% What are they doing?
%Comparing ARIMA and ETS to eachother on a univariate retail time series.
%% Why are they doing it?
%They aim to test two ETS and ARIMA on the consumer retail forecasting domain. 
%Which is a difficult domain due to strong trend and seasonal patterns.
%Comapre the forecasting perfomance of ETS and ARIMA models when
%applied to a case study of retail sales of five different categories of woman footwar
%from the Portuguese retailer Foreva.

% Skrive om. Finne kilde
[Forecasting aggregate retail sales: a comparison of ANN and traditional methods. I. ALON]
Winters exponential smoothing and ARIMA perform well when macro-economic
conditions are relatively stable. When economic conditions are volatile
ANNs outperform linear methods and multi-step regression is preferred.

% Skrive om. Finne kilde
[Zhang and Qi...] and [Kuvalmaz et al.]
Investigated the use of ANNs in forecasting time series with a strong trend and seasonality.
They concluded that the overall out-of-sample forecasting perfomance of ANNs, 
evaluated via RMSE, MAE and MAPE is not better than ARIMA models in predicting
retail sales without appropriate data preprocessing,
namely detrending and deseasonalization.

[Pan et al.]
Tested a hybrid forecasting method of integrating MED and NN and compares them with direct NN models.
and vs the ARIMA model for aggregate retail sales forecasting.
Different macroeconomic conditions were studied.
Hybrid NN model is more stable compared to direct NN and arima during volatile economy.
ARIMA performs consistently well during stable economic activity.

% This section should describe earlier work done within the field of E-commerce and Sales prediciton!
% This is the main explainer for why we need do any work, and why we cant just apply a model.
% What can be done to improve the models and take the most important pars from the different approaches.
% This will more or less be what i though motivation was going to become!
% Thus; Se Motivation_prev.tex for reference

\cite{Bowen2020} ....