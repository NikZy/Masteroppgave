
\section{Hybrid Network for Time series forecasting}
\label{section:RelatedWork:Hybrid}

% TODO! -> Check that this is actually a hybrid model! What is a hybrid model! And is it CNN + LSTM?
% CNN = Feature extraction
% LSTM = Sequence learning

% Introduce hybrid methods and why they are a thing.
% -> Write about the ARIMA and BP Hybrid model.
% Introduce the two new methods
% -> CNN-AE and LSTM
% -> CNN and LSTM-AE

Improvements in time series forecsating has emerged the last few years with the introduction of deep learning.
Previous state if the art methods, such as the ARIMA model, has been exchanged for deep learning methods using convolution and recurrent networks to improve accuracy.
One such improvement strategy has been to introduce hybrid models.
These models are comprised of different models with uniqe abilities, helping to increase the accuracy of the connected model.


One example of such a hybrid model is the ARIMA-BP model explored in [Add Citation!].
This paper explores the predictive abilities of the ARIMA model and a Back-Propagation Neural network on a time series problem of sales forecasting.
The ARIMA and BP models were applied to the problem, with the ARIMA somewhat outperforming the BP model.
Howerver, by combining the two models, the hybrid model was able to acchieve a lower predicitve accuracy than any of the two models on their own.
This hybrid model is heavily based on the state of the art statistical method of ARIMA.
However, as we have explored in [Add referance to "Statistical methods vs Neural Networks"], the introduction of more advanced neural networks often achieve better predicitive results than the old state of the art ARIMA model.
We therefore argue that it should make sense to look further into other hybrid models, utilizing these new and improved models.
By combining methods such as Convolution and Recurrent networks in a hybrid model, it should yeald a better result than they would on their own.
% ARIMA-BP er en hybrid model som gir bedre resultater enn de enktelte delene alene. Gir et godt grunnlag for at hybrid modeller er bra!



This propsed combination of models is explored in a few different ways.
One such hybrid model is the "CNN and LSTM-Autoencoder" method explored in \cite{Khan2020}.
This method was used to explore the predictive ability of the hybrid model on electricity forecasting in residential and comercial buildings.
The findings of the paper concludes with the same result as we propose in above.
The afformentioned method is a hybrid framework based on a convolutional networs, LSTM and Autoencoder.
Convolution was used to extract features from the input data, before the LSTM-Autoencoder attempt to extract temporal dependencies between the sequences.
The propsed method was applied to the electricity forecasting problem alongside other models such as ARMA, SVM, SVR and others.
The framework outperformed the state-of-the-art models at prediction using several different performance metrics such as MSE, MAE RMSE and MAPE.


Similar to the framework described above, yet another method utilizing the same methods was defined in \cite{Khan2020}.
% TODO! Add correct citation!!!!
This model differs in the selected architecture, creating a Convolutional Autoencoder insted of the previous LSTM Autoencoder.
The Convolutional Autoencoder is used to extract ...







% Snakke om at LSTM og Convolusjon er gode modeller som har vist gode resultater tidliger. Her kan vi refferere til related work seksjonen som kommer over.
% NB: Denne related work biten er ikke skrevet ennå, men bir det. Så ta forbehold om at noe endrer seg.



% Dette er helt nye modeller og viser seg å være bedre enn tidligere state of the art, samt current state of the art LSTM (i guess)
% Forklare at denne nye metoden som tidliger er brukt til tids-serie analyse, kan anvendes i vårt domene og burde fungere bra.
% Sette dette sammen med gruppering av data blant annet, refferer tilbake til der det skrives i Related work, og forklare at det detfor kan anvendes her.

% Add the argument or grouping here! Adding this should give better predictions and hopefully find connections between data and trends.
% This is only a hunch, and something the domain expert (Prisguiden) thought could be exploited!

