
\section{Model Structure}
\label{section:RelatedWork:Model-structure}
Since our problem domain consists of multiple product categories and sub-categories,
and we want to forecast all these categories,
we have multiple possible ways to construct a forecasting model.

A simple approach would be to make one model for each product category.
One could assume that each category of products is independent of the other.
This assumption will probably not hold water for all cases. People who buy shaving cream probably also buy a razer, as shaving cream will not have much use of its own.
An alternative is to build a more complex model, that looks at a bigger picture.
This section will explore different state-of-the-art model structure approaches.

% TODO: Introduksjon om model struktur...

% En sub-section for Univariate vs Multivariate tidligere arbeid

% En sub-section for Local vs Global

% En sub-section som fokuserer på diskusjon rundt de ulike kombinasjonene og hva som er relevant i vår oppgave / problemstilling
\subsection{Global vs Local methods}
On the topic of having to forecast many time series as a group \cite{Montero-Manso2021} paper has a good overview.
The paper points to two big disadvantages for univariate time series on a cluster of series.
The number one shortcomming is sample size. The second is scalability.
Scalability is a problem when you have a group of time series as each series require a seperate model
that requires human intervention. Forecasting a cluster of time series in this manner is called
\textit{the local approach.}

A univariate alternative to a local approach is \textit{the global approach}
\citep{Rabanser2020}.
The global approach works by pooling the data of all series togheter and fit a single univariate forecasting function. It prevents over-fitting because of the larger sample size.
% source slainas, flunkert, gasthaus & Januschowski 2020
The global approach has been introduced to exploit the natural scenario where all series
in the set are similar or related. An example given by the authors are the demand of fictional
books follows a similar pattern for all subgenres, stores or cover designs.
The idea behind is the strong assumption that all the time series in the set
come from the same process.

This exact method was used by \cite{Bandara2017}. However they argue that 
when building global models for a time-series database, the models are
potentially trained across disparate series, which may be detrimental to the overall accuracy.
They suggest to build separate models for subgroups of time series.
These groups can be based on domain knowledge, which proved to be the best option. With absence of 
domain knowledge they propose an automatic grouping mechanism to cluster series togheter.
Their method achieves consistent improvements over a the baseline LSTM model.
And conclude that expoiting similarities of multiple time series in one model is a competitiv method.

%The paper tests their model on two competition datasets, and achieves competitive results.
%On the CIF2016 dataset, the model outperforms all other models.
%On the NN5 competition it ranks 6th overall, and achieves consistent improvements over the baseline LSTM model.
%They conclude that exploiting similarities of multiple time series in one model
%is a competitive method.


%As recent studies show puzzlingly good performance of time series that cannot be considered related.

\cite{Rabanser2020} show that even if the strong assumption that all the time series analysed by a global
method is generated by the same process, is false, the global method will pay off in forecasting accuracy.
The paper argues that global and local methods for forecasting
sets of time series are equally general. The global method is neither restrictive nor requires
similarity or relatedness in the set.
But they point out that generalization of global models assumes groups of independent time series.
Under heavy dependence, global models lose their beneficial perfomance guarantees. 

The paper \cite{Hewamalage2021} comes to the same conclusion. Stating that 
even on datasets that involve many heterogeneous series, the strong modelling capabilities of RNNs can drive
them to perform competitevely in terms of forecasting accuracy.
\todo[]{Skrive om paperet hvor klustering er gjort på tidsserier}

\subsection{Univariate vs Multivariate time series}
%In the previes section we discussed how statistical methods can outperform ML methods on
%time series without a large enough sample size.
\cite{Bandara2017} points out that statistical methods, like ARIMA, are bound to
univariate time series. In the world of Big Data and lots of time series that correlate with each other,
treating each time series separately and forecasting each in isolation might miss the big picture.
The paper argues that the ability to make models that can be trained globally across all series
holds a competitive advantage over models like ARIMA and ETS.
A model like this would be a multi-input-multi-output model, which forecasts multiple time series
at the same time.

The paper \cite{Rabanser2020} has some good arguments when comparing univariate vs multivariate models.
Both multivariate and global univariate methods work on groups of time series, but global methods
has the advantage of being more applicable because it does not require observations of multiple
time series at the time of forecasting.
Also, multivariate time series models work on groups that are supposed to have some form of 
dependence between them, while global models work on any group.
But when there exists such a dependency the global method will not capture it directly.
\cite{Hewamalage2021} states in their \textit{7. Future directions} chapter that complex
forecasting scenarios, such as a retail sales forecast, the sales of different products
may be interdependent.
Forecasting in such a context may require a multivariate model.

% What did they do?
\cite{Laptev}
Wanted to make a single time series model to
accurately make time series predictions during special events.
Extreme event prediction depends on numerous external factors,
that can include weather, city population growth or marketing
changes.
They propose a global, multivariate, autoencoder, LSTM network.
Their results are promising. They outperform their existing
proprietary model by up to 18\% on Uber data.
They also show the models generalization power, by training on Uber data
but then testing it on the public M3 dataset, where they achieve an above average result.

% Discussion
In \cite{Laptev} discussion they point out three scriteria for 
choosing a neural network over a statistical method:
(a) number of time series to model are high, (b) Length of the 
times series are high, and (c) correlation among the time series.
Our problem domain meets all these criteria. 

%% Statistical vs nn
%\cite{Laptev}  points out that 
%%[ original source: Ye & Keogh, 2009 ]
%classical statistical time-series models usually require manual tuning
%to set seasonality and other parameters.
%A LSTM will 


% What results did they get?


\subsection{Discussion}
A local univariate approach would be similar to how an ARIMA model would work.
It would have the benefit of being simple, but in order to forecast $n$ product categories, 
on would need $n$ models.
The benefit of having $n$ models is that consumer product categories are a dynamic set, constantly changing.
When a new category of products are created, a forecasting model are created with it. It's forecasting ability
would be limited, but would gradually improve as more data would accumulate.

The greatest drawback of a local univariate approach would be it's the assumption that all product categories
are indendent. We can expect some groups to correlate with each other. These interdependent relationships
could greatly improve forecasting ability.

A multivariate approach will capture these relationships.
However, one needs a thorough understanding of the underlying data, in order to build meaningful models.
A thorough covariance analysis to identify correlating groups of product categories would be best.
However, this might be outside of scope.

A multivariate model do have some drawbacks on a ever-changing product category domain.
First, its the issue that not alle categories has the same life of existens, and thus a variable
amound of history data, which would amount to a lot of null values.
Second, each time a category is added, or changed a covariance analysis has to be done in order to identify
which covariance group the category belongs to. Then that model has to be retrained.
Depending on the amount of changes to the category tree, this might be a hindrance.

The univariate global approach has the benefit of overcoming many of the hindrances of a multivariate model.
As it do not care which product category you feed it, it is very scalable. The argument that \cite{Bandara2017}
give, that a single global model might be detrimental to overall accurecy, could be overcome by clustering product categories together. This is true even if \cite{Rabanser2020}
shows that global models will improve accuracy, even if the global model assumption is not satisfied.

How such a clustering of product categories is to be made is another question.
\cite{Bandara2017} suggests a K-means clustering technuiqe
which looks at several time series characteristics, such as 
strength of trend, strength of spikiniess, strength of curvature,
and sales quantile.

The global method will not capture interdependent relationships directly,
but might do so indirectly. It also has the potential to build a lot more complicated model,
as the amount of traning data will drasticly increase, which again will make the model less prone to 
overfitting.

% Should this be under multi vs univariate section?
% Currently it does not have any references
One last approach is to build a multivariate, multi-input-single output model which only forecasts a single product category
at the time. The input to such a model might be a decomposition of the target time series. 
For example decomposing a time series into a component that show how the trend of the series behaves over time.
One component for the cycles, and one for noise.
Another option is to rely on external data sources which correlate with the target series.
For example it is to be expected that if a product category group suddenly spikes in interest at Prisguiden.no,
then a similar spike would happen at Google.
Or bad weather might have an effect on interest rates at Prisguiden because people are inside on the computer more.
Such a multi-input-single-output model can also have the additional benefit of being a global model.
This is exaclty what \cite{Laptev} did, and they achieved a promising result.


%This apprach would also be hard to scale. If one 

% Global models
% + scalable
% + Can use complicated models because of a lot of training data
% + Has seen some promesing results resently
% - Might miss interdependent relationships
% - A single global model might be too general
% Could split into sub groups

% Multivariate models
% + Can see interdependent relationships between time series
% - Difficult to scale, as adding one category means changing the models 
% shape
% Will maximize the effectiveness of CNN?



%\subsubsection{Limitations of statistical methods}
%\todo[inline]{Skrive om limitations til statistiske metoder and
%* Univariate
%* stationary time series
%* Dealing with extreme values
%https://towardsdatascience.com/limitations-of-arima-dealing-with-outliers-30cc0c6ddf33
%}
%Pro ML: size matters \cite{Cerqueira2019}
%
%From \cite{Guen2019}:
%Traditional methods for time series forecasting include linear autoregressive models, such as
%ARIMA odel, and exponential smoothing, which both fall into the broad category of of linear 
%state space models (SSMs). These methods handle linear
%dynamics and stationary time series (or made stationary by temporal differences).
%However the stationarity assumption is not satisfied for many real world tmie series that can present
%abrupt changes of distribution...