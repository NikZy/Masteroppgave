\section{Model Structure}
\label{section:RelatedWork:Model-structure}
The structure of the forecasting model is an important aspect when the goal is to forecast
multiple time-series.
A simple approach would be to make one model for each time-series.
One could assume that each series are independent, but this assumption will probably not hold water for all domains.
For example, in the E-commerce domain,
people who buy shaving cream probably also buy a razer, as shaving cream will not have much use of its own.
An alternative is to build a more complex model that looks at a bigger picture.
This section will explore different state-of-the-art model structure approaches.

\subsection{Global versus Local methods}
On the topic of having to forecast many time-series as a group, the paper from \cite{Montero-Manso2021} provides a good overview.
The article points to two significant disadvantages for univariate models on a cluster of series.
The number one shortcoming is the sample size. The second is scalability.
Scalability is a problem when you have a group of time-series as each series requires a separate model
that requires human intervention. Forecasting a cluster of time-series in this manner is called
\textit{the local approach.}

A univariate alternative to a local approach is \textit{the global approach}
\citep{Rabanser2020}.
The global approach works by pooling all series data together, fitting a single univariate forecasting function. It prevents over-fitting because of the larger sample size.
% source slainas, flunkert, gasthaus & Januschowski 2020
The global method has been introduced to exploit the natural scenario where all series
in the set are similar or related. An example given by the authors is the demand for fictional
books follows a similar pattern for all subgenres, stores, or cover designs.
The idea behind this is the strong assumption that all the time-series in the set
come from the same process.

This exact method was used by \cite{Bandara2017}.
Their domain has similarities to ours. They want to forecast a database of E-commerce time-series
from Walmart.com.
They argue that when building global models for a time-series database, the models are
potentially trained across disparate series, which may be detrimental to the overall accuracy.
They suggest building separate models for subgroups of time-series.
These groups can be based on domain knowledge, which proved to be the best option. With the absence of
domain knowledge, they propose an automatic grouping mechanism to cluster series together.
Their method achieves consistent improvements over the baseline LSTM model.
And conclude that exploiting similarities of multiple time-series in one model is a competitive method.

%The paper tests their model on two competition datasets, and achieves competitive results.
%On the CIF2016 dataset, the model outperforms all other models.
%On the NN5 competition it ranks 6th overall, and achieves consistent improvements over the baseline LSTM model.
%They conclude that exploiting similarities of multiple time-series in one model
%is a competitive method.


%As recent studies show puzzlingly good performance of time-series that cannot be considered related.

\cite{Rabanser2020} show that even if the strong assumption that the same process generates all the time-series analyzed by a global
the method is false; the global method will pay off in forecasting accuracy.
The paper argues that global and local methods for forecasting
sets of time-series are equally general. The global method is neither restrictive nor requires
similarity or relatedness in the set.
But they point out that generalization of global models assumes groups of independent time-series.
Under heavy dependence, global models lose their beneficial performance guarantees.

The paper \cite{Hewamalage2021} comes to the same conclusion. Stating that
even on datasets that involve many heterogeneous series, the strong modeling capabilities of RNNs can drive
them to perform competitively in forecasting accuracy.
%\todo[]{Skrive om paperet hvor klustering er gjort på tidsserier}

\subsection{Univariate or Multivariate time-series}
%In the previes section we discussed how statistical methods can outperform ML methods on
%time-series without a large enough sample size.
\cite{Bandara2017} points out that statistical methods, like ARIMA, are bound to
univariate time-series. In the world of Big Data and lots of time-series that correlate with each other,
treating each time-series separately and forecasting each in isolation might miss the big picture.
The paper argues that the ability to make models that can be trained globally across all series
holds a competitive advantage over models like ARIMA and ETS.
Such a model would simultaneously use multiple time-series as input to predict future values for the time-series.

The paper \cite{Rabanser2020} has some good arguments when comparing univariate versus multivariate models.
Both multivariate and global univariate methods work on groups of time-series, but global methods
have the advantage of being more applicable because it does not require observations of multiple
time-series at the time of forecasting.
Also, multivariate time-series models work on groups that are supposed to have some form of
dependence between them, while global models work on any group.
But when such a dependency exists, the global method will not capture it directly.
\cite{Hewamalage2021} states in their \textit{7. Future directions} chapter that complex
forecasting scenarios, such as a retail sales forecast, the sales of different products
may be interdependent.
Forecasting in such a context may require a multivariate model.

% What did they do?
\cite{Laptev}
Wanted to make a single time-series model to
accurately make time-series predictions during special events.
Extreme event prediction depends on numerous external factors, including weather, city population growth, or marketing
changes.
They propose a global, multivariate, autoencoder, LSTM network.
Their results are promising. They outperform their existing
proprietary model by up to 18\% on Uber data.
They also show the model's generalization power by training on Uber data
but then testing it on the public M3 dataset, where they achieve an above-average result.
% Discussion
In \cite{Laptev} discussion, they point out three criteria for
choosing a neural network over a statistical method:
(a) number of time-series to the model is high, (b) Length of the
times series are high, and (c) correlation among the time-series.
Our problem domain meets all these criteria.
The third (c) criteria are worth mentioning because it directly
contradicts the argument proposed by \cite{Montero-Manso2021}%\cite{Hewamalage2021}
which states that in a set of interdependent time-series,
a global model lose their beneficial performance guarantees,
because a global model assumes independence.
This again contradicts \cite{Hewamalage2021}, which explicitly states
retail forecasts, which, as argued, is similar to E-commerce,
as a perfect example for a global model.
Of all the papers we found, none of them concluded that a global method
on time-series that have some form of interdependency would directly hurt the model.


%Our problem is similar to the DeepGLO model made by \cite{Sen2019}. 

\cite{Sen2019} proposes a method to handle predictions of thousands of interdependent correlating time-series.
Their proposed method is largely based on two components. The first component is based on the work of
\cite{Yu}, which proposes a Temporal regularized matrix factorization (TRM) for high-dimensional time-series prediction.
The idea is that the TRM can look at all the time-series and capture the global patterns during prediction.
The TRM can supposedly handle as much as 50.000 time-series.
%The drawback of the TRM is that it only captures linear dependencies between the series.
% In order to capture non-linear depndencies 
The output of the global model is used as a covariate to a final, local temporal convolution network. This final model
will then focus on local per time-series properties, as well as properties from the global dataset.



%% Statistical vs nn
%\cite{Laptev}  points out that 
%%[ original source: Ye & Keogh, 2009 ]
%classical statistical time-series models usually require manual tuning
%to set seasonality and other parameters.
%A LSTM will 


%\subsubsection{Limitations of statistical methods}
%\todo[inline]{Skrive om limitations til statistiske metoder and
%* Univariate
%* stationary time-series
%* Dealing with extreme values
%https://towardsdatascience.com/limitations-of-arima-dealing-with-outliers-30cc0c6ddf33
%}
%Pro ML: size matters \cite{Cerqueira2019}
%
%From \cite{Guen2019}:
%Traditional methods for time-series forecasting include linear autoregressive models, such as
%ARIMA odel, and exponential smoothing, which both fall into the broad category of of linear 
%state space models (SSMs). These methods handle linear
%dynamics and stationary time-series (or made stationary by temporal differences).
%However the stationarity assumption is not satisfied for many real world tmie series that can present
%abrupt changes of distribution...