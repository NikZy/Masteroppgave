% https://ieeexplore.ieee.org/stamp/stamp.jsp?tp=&arnumber=8996842&tag=1

\section{Convolutional Autoencoder and LSTM}

% This is a new type of time series forecasting method
% Other more simple methods are also used.
% LSTM is one of them, as well as some basic ARIMA and SVR (Suport vector regression?)

In an efort to increase the predictive accuracy and decrese the error metric in time series prediction,
new predictive methods are devised and tested.

One such method is proposed in the paper \cite{Khan2020}.
\todo[inline]{This is the wrong citation. Add the correct one.}
In order to decrease the predictive error, this paper explores the use of a Convolutional Autoencoder and LSTM.
Conducting experiments using this method, in comparison with a standard LSTM model,
the paper found that the propsed method resulted in predictions with lower error, using MAPE and RMSE as error metrics.
\todo[inline]{Add links to MAPE and RMSE from the Background and Theory chapter.}
Using datasets containing information about traffic flow, as well as factory equipment fault information,
the proposed method clearly outperformes the LSTM method.

The proposed method was created in order to properly deal with time series with high noise and fluctuations.
Due to the feature extraction ability of the convolutional network,
and the ability of the autoencoder to ignore noise data, the proposed model should be well suited for time series prediction with high noise data.


The paper also adds a comparison of the proposed method against and ARIMA model, and a SVR model, in addition to the LSTM model.
This shows a clear difference in the predicitve error of the different methods, in favor of the CNN-AE with LSTM.
The propsed method is much better at processing the higly fluctuating data features in the propsed datasets.
% TODO -> Last sentence = Meh

% TODO? -> This is only a univariate method. We were thinking a global multivariate method, something that is quite a lot different

% TODO -> Should we describe here that we want to use this method in our paper? Or should this be in the model-arch / discussion part of the paper?
%       -> Perhaps is should be mentioned in the Results section, considering this is the result of the paper and literature analysis?

% -> What is this missing? It is clear that it does not contain all that we are going to use, say grouping of data and several time series, so this should be mentioned as why this is related work.

% -> Det er viktig 


%%%%%%%%%%%%%%%%%%%%%%%%%%%%%%%%%%%%%%%%%%%%%%%%%%%%%%%%%%%%%%%%%%%%%%%%%%%%%%%%%%%%%%%%%%%%%%%%%%%%%%%%%%
%%%%%%%%%%%%%%%%%%%%%%%%%%%%%%%%%%%%%%%%%%%%%%%%%%%%%%%%%%%%%%%%%%%%%%%%%%%%%%%%%%%%%%%%%%%%%%%%%%%%%%%%%%
%%%%%%%%%%%%%%%%%%%%%%%%%%%%%%%%%%%%%%%%%%%%%%%%%%%%%%%%%%%%%%%%%%%%%%%%%%%%%%%%%%%%%%%%%%%%%%%%%%%%%%%%%%
%%%%%%%%%%%%%%%%%%%%%%%%%%%%%%%%%%%%%%%%%%%%%%%%%%%%%%%%%%%%%%%%%%%%%%%%%%%%%%%%%%%%%%%%%%%%%%%%%%%%%%%%%%

% -> Det er viktigst å få frem hva som er grunnen til at denne oppgaven gjennomføres
% -> Det må være et problem.
% -> Ja, en kan referere til tidligere arbeid, men det er ikke det som er hovedfokuset her. Fokuser heller på arbeid som allerede er vel etablert for å si noe om oppgaven.
% -> Gjør det tydelig at det faktisk er et problem som skal løses, men du trenger ikke sammenligne masse ulike methoder som har vært brukt tidligere
% -> Hva som er gjort tidligere og nylig er mye mer relevant for "Related work".

% -> I related work, gå etter kategori, ikke etter en og en paper.
% -> Det er ikke noen grunn til å bare gjennfortelle det om er gjort før! Det kan være 3 setninger, eller et avsnitt. Spiller ingen rolle
% -> Det viktigste her er at det som nevnes settes i en kontekst! Ikke bare hva som er gjort.
% -> En overskrift blir f.eks "Sales prediction" hvor vi nevner det som er gjort nylig og snakker om hvorfor det er grun til å prøve noe annet!.
% -> Her er det viktig at vi gjør noe som gir noen nytteverdi! Vi skal komme med noe "nytt" og klare å reflekter. Ikke bare gjennfortelle


