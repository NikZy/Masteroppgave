\section{Statistical methods versus Neural Nets}
\label{section:RelatedWork:Statistical-NN}

Statistical methods like ARIMA and exponential smoothing have long been state-of-the-art for forecasting time-series.
In recent years Deep Neural Networks have gotten more attention.
\cite{Makridakis2018} wanted to objectively test these up-and-coming machine learning models with the old state-of-the-art methods.
Performance was evaluated across multiple forecasting horizons using a large subset of 1045 monthly time-series.
Among the evaluated machine learning methods are:
\textit{Bayesian Neural Network (BNN)},
\textit{Multi-Layer Perceptron (MLP)},
\textit{K-Nearest Neighbor regression (KNN)},
\textit{Recurrent Neural Network (RNN)}, and
\textit{Long Short Term Memory neural network (LSTM)}

%If a time-series is stationary, then using its statistical properties is shown to be an effective and computationally cheap method for forecasting
%\cite{Makridakis2018}.
%In a paper written by \citeauthor{Makridakis2018} statistical methods are tested against machine learning methods for
%forecasting time-series.
%Performance is evaluated across multiple forecasting horizons using a large subset of 1045 monthly time-series.
%Among the evaluated machine learning methods are

% \begin{itemize}
%   \item \textit{Bayesian Neural Network (BNN)}
%   \item \textit{Multi-Layer Perceptron (MLP)}
%   \item \textit{K-Nearest Neighbor regression (KNN)}
%   \item \textit{Recurrent Neural Network (RNN)} and
%   \item \textit{Long Short Term Memory neural network (LSTM)}.
% \end{itemize}

Among the statistical methods evaluated are:
\textit{Naive2},
\textit{ETS}, and
\textit{ARIMA}.
%The methods are evaluated with different loss functions and metrics, with simple one-step-ahead and multiple steps ahead methods.
% \begin{itemize}
%   \item \textit{Naive2}
%   \item \textit{ETS}
%   \item \textit{ARIMA}
% \end{itemize}
They conclude that the statistical methods outperformed all ML methods in terms of accuracy, measured for both
simple one-step-ahead and multiple steps ahead methods.
% The statistical methods also had less model complexity and computational cost for training.
% The best performing model was ETS with a sMAPE score of 7.12.
% The second-best model was ARIMA, with a score of 7.19.
% The third worst model was LSTM, with a score of 11.67.

The paper \cite{Makridakis2018} highlights some of the drawbacks of ML methods. Especially their complexity, their computational cost,
their lack of explainability and their inability to show certainty in their predictions.
However, the study has gotten some well-defined critic. The authors \citeauthor*{Cerqueira2019} points
out that \cite{Makridakis2018} does their experiments on datasets of too-small sample size.
Their largest time-series sample size among the 1045 datasets is 144 points, while the smallest contains 118.
They hypothesize that these datasets are too small for an ML method to generalize properly.

\cite{Cerqueira2019} conduct a similar study on 90 univariate time-series, in which
all the datasets have a sample size above 1000.
Statistical methods are evaluated against machine learning methods at different sample sizes,
in order to test whether the sample size has an impact on performance.

The paper \cite{Cerqueira2019} concludes that sample size impacts machine learning methods drastically.
Statistical methods outperformed ML methods up to around a sample size of 130 data points.
In cases with larger sample sets, the machine learning methods were generally shown to outperform the statistical methods.

% \iffalse
%   % Dette er mye mer relevant for diskusjon. Dette nevnes senere.
%   This is good news for us, as our datasets do have a sample size of well above 660 at this moment.
%   Depending on when we conduct the experiment, the number will grow to the range of 1000 to 1200,
%   as data is being gathered every day as we speak.
%   \todo[inline]{Should this be described when we don't describe out data? Should have section with data information if we are to inluce this!}

%   One drawback of the \cite{Makridakis2018} study is that it does not test any of our chosen ML methods.
%   They tested:
%   \begin{itemize}
%     \item \textit{A rule-based model (RBR)}
%     \item \textit{A random Forest method (RF)}
%     \item \textit{Guassion Process regression (GP)}
%     \item \textit{The Multivariate adative regression (MARS)}
%     \item \textit{Generalized linear model (GLM)}
%   \end{itemize}
% \fi

From the discussion originating with the two papers described above,
it should be clear that machine learning methods are to be preferred in the event of sufficient data.
\cite{Hewamalage2021} confirms this statement. They further conclude that even with the higher computational costs
associated with ANNs, such computational costs are feasible because of the general availability of cloud computing.
% With this in mind, we take into consideration the work done in \cite{Hewamalage2021}.
% Here a thorough analysis is done of the current state-of-the-art recurrent neural networks, such as the LSTM.
Complex deep learning methods now have benefits over simpler statistical methods in many forecasting cases.
Although such statistical methods are reliable and produce a relevant explainable predictive baseline,
modern deep learning methods can accomplish higher predictive accuracy.

The presented papers show the relevance of exploring new and more complex methods to improve time-series prediction.
Deep learning methods such as LSTMs have high predictive accuracy,
and exploring these methods further may prove to be advantageous.
Nevertheless, statistical methods supply a well-defined predictive baseline for a time-series.
Additionally, statistical methods have the advantage of improved explainability,
making these methods more readily understandable.
With this in mind, using well-defined statistical methods as a baseline prediction to evaluate new predictive methods should be considered.
Only comparing highly complex deep learning methods may prove difficult if they cannot make accurate predictions.
A baseline should therefore be a meaningful way of evaluating more complex methods.

\iffalse
  % --- Kommentert ut ---- Refererer veldig til hva som gjøres videre. Sammenligning av Related work som dette hører mer hjemme i discussion under konklusjon
  The paper \cite{Bandara2017} points out that in non-stationary time-series, the distant past is typically less
  useful for forecast, as underlying patterns and relationships will have changed in the meantime.
  % TODO: This hopefully not apply to us as our time seires do seem to have consistency

  In \Cref{section:RelatedWork:forecasting-ecommerce}
  we highlighted ANNs ability to outperform statistical methods when macro-economic
  conditions were unstable.
  In \Cref{section:RelatedWork:Model-structure} we mentioned how statistical models
  potentially missed the bigger picture in a domain of many related time-series.
  These findings makes it clear to us that limiting our solution to a statistical method
  will be a mistake, and exploring a more complicated ANN model seems beneficial.
  However, a need for a baseline model in order to compare the results of the ANN model,
  a statistical method, like SARIMA could be useful.
\fi
