
\section{Hybrid frameworks for Time series forecasting}
\label{section:RelatedWork:Hybrid}

Improvements in time series forecasting have emerged in the last few years with the introduction of deep learning.
Previous state-of-the-art methods, such as the ARIMA model, have been exchanged for deep learning methods using convolution and recurrent networks to improve accuracy.
One such improvement strategy has been to introduce hybrid models.
These models are comprised of different models with unique abilities, helping to increase the accuracy of the connected model.


One example of such a hybrid model is the ARIMA-BP model explored in \cite{Bowen2020}.
This paper explores the predictive abilities of the ARIMA model and a Back-Propagation Neural network on a time series problem of sales forecasting.
The ARIMA and BP models were applied to the problem, with the ARIMA somewhat outperforming the BP model.
However, by combining the two models, the hybrid model was able to achieve higher predictive accuracy than any of the two models on their own.
This hybrid model is heavily based on the state-of-the-art statistical method of ARIMA.
As we have explored in \ref{section:RelatedWork:Statistical-NN}, the introduction of more advanced neural networks often achieves better predictive results than the old state-of-the-art ARIMA model.
We, therefore, argue that it should make sense to look further into other hybrid models, utilizing these new and improved models.
By combining methods such as Convolution and Recurrent networks in a hybrid model, it should yield a better result than they would on their own.
% ARIMA-BP er en hybrid model som gir bedre resultater enn de enktelte delene alene. Gir et godt grunnlag for at hybrid modeller er bra!

% TODO -> Refference to when LSTM or LSTM-Autoencoder works well!
% TODO -> Discuss why it makes sense for a hybrid model!
% TODO -> Make it more clear that LSTM is also explored in the Hybrid models, but it is not as well suited.

This combination of models is explored in a few different ways.
One such hybrid model is the "CNN and LSTM-Autoencoder" method explored in \cite{Khan2020}.
This method is used to explore the predictive ability of the hybrid model on electricity forecasting in residential and commercial buildings.
The findings of the paper conclude with the same result as we propose above.
The aforementioned method is a hybrid framework based on convolution, LSTM, and Autoencoder.
Convolution is used to extract features from the input data before the LSTM-Autoencoder extract temporal dependencies between the sequences.
The proposed method is applied to the electricity forecasting problem alongside other models such as ARMA, SVM, SVR, and others.
The framework outperformed the state-of-the-art models at prediction using several different performance metrics such as MSE, MAE, RMSE, and MAPE,
as well as other hybrid deep learning methods such as a CNN-LSTM.


Similar to the framework described above, yet another method utilizing the same methods was defined in \cite{Zaho2019}.
This model differs in the selected architecture, creating a Convolutional Autoencoder instead of the previous LSTM Autoencoder.
The Convolutional Autoencoder is used to extract and reduce dimensional features before the LSTM extracts the temporal features.
Similar to the previously mentioned method, this "CNN-AE LSTM" hybrid method is tested against several other current predictive models.
The proposed method outperforms methods such as LSTM, ARIMA, and SVR, achieving a much lower predictive error than the other methods.


\ref{section:RelatedWork:Statistical-NN} explores the predictive ability of statistical methods in comparison to deep learning methods.
This section discusses how deep learning methods such as LSTM and Convolution work well in order to predict in cases where there is enough data.
Thus, the usage of deep learning methods such as LSTM and Convolution is to be preferred when enough data is attained.
The proposed hybrid frameworks outperform deep learning methods such as LSTM.
With this in mind, it is clear that a hybrid method is well suited for predictions in our problem space.
A hybrid framework has already been shown to exceed the predictive capabilities of its individual parts.
This was done with the ARIMA-BP model explored in \cite{Bowen2020}, showing that the connected hybrid method performed better than the ARIMA or BP models could on their own.
\cite{Zhao2019} also concludes with this, achieving much lower predictive error with the propsed method in comparsion to a pure LSTM network.

% TODO -> Ser ikke helt vitsen med denne seksjonen annet enn å legge til flere kilder
% https://dl.acm.org/doi/10.1145/3460179.3460184
% https://ieeexplore.ieee.org/document/8947933
Other methods attempting to improve predictive forecasting has also shown good results.
Frameworks such as with the proposed AE-LSTM from \cite{VanHoa2021} has shown the use of an autoencoder to increase the accuracy of the predictions over a basic LSTM method.
This paper focus on forign exchange rate forecasting, and shows the increased accuracy achieved using a Autoencoder and LSTM, over a basic CNN or LSTM network.
Similarly, \cite{Zhang2020} use a Gated Dialated Causal Convolutional Encoder-Decoder in network traffic forecasting.
It shows the decrese in predictive error with the use of a advanced Convolutional Autoencoder in comparison to a deep learning method such as an LSTM.
With this in mind, we should consider more complex models than basic LSTM and CNN networks for increase predictive accuracy.
Based on this, we propose that a hybrid model would achieve higher predictive accuracy than the individual parts that it is comprised of.
Despite the LSTM beeing a well suited method for forecsating prioblems, it is clear that it can be improved upon in several different ways.
We propose that a framework using a Convoltional-Autoencoder and LSTM would achieve better accuracy than either of its parts could alone.

% Hybrid networks
\cite{Laptev} found that a vanilla LSTM-network performed worse than
state-of-the-art statistical univariate models. But adding
an autoencoder for feature extraction to the network achieved better results


% It is likely that this would also be the case for a hybrid CNN and LSTM model.


The proposed problem-space has data with high fluctuations and noise.
In order to increase the predictive abilities of a model, a method well suited for working with data with high noise should be selected.
A CNN-AE model should be able to solve this problem.
The CNN is able to extract the spatial features of the data while the AE can filter out the noise and fluctuations in the data.
By then adding a LSTM network at the end, the model should be able to extract the temoral featrues from the data.
This hybrid framework should therefore be well suited for the task at hand.



%The proposed methods for hybrid networks using CNN, Autoencoder and LSTM, has shown to excide the predicite capabilities of methods such as ARIMA and LSTM.

% TODO: Jeg savner kilder til papers som har fått til time series forecasting med 
% LSTM, og med autoencoder, og CNN så kan vi dra argumentet at alle sammen 
% bare må bli enda bedre!
