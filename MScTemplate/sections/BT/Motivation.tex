\section{Motivation}
\label{section:BT:Motivation}

With the emergence of the internet, large parts of the human experience are conducted online.
Online services supply users with everything from entertainment and social media, to banking and online shopping.
The accessibility of online services such as online retailers has enabled users to shop for most products online.
In relation to online shopping, competitive pricing emerged.
When shopping online, a user will often consider the pricing of the product to be purchased, often comparing the retail prices between different retailers.
%This concept created the basis for Prisguiden to be created.


In order to easily and effectively compare prices of products amongst retailers, 
\textit{Prisguiden} was created.
Products and product categories were introduced to the site, collecting the pricing information of the products from several different retailers.
Different product categories have different retailers, and thus the collection of the information is somewhat distributed.
When new products hit the market, these products are introduced to the product portfolio, in order for uses to compare the prices.
Operating such an online service has enabled \textit{Prisguden} to accumulate user data such as product and product category interest, and more.
This is highly fluctuating time series data, regarding user trends and use patterns, yet the data is not used for anything.
The collected time series data could hold a lot of relevant information.
\textit{Prisguiden} intend to use this data in order to predict future product and category trends.
With this information, they would be able to effectively know where to put their resources in regards to introducing and updating product listings and pricing for product categories 
that are on the rice in popularity.
\linebreak

% Prisguiden was created to check prices of products between retailers
% It has made it easies to compare prices.
% Prisguiden is thus able to collect a lot of data about what products and product categories are searched and thus popular
% This creates a time series of data



In order to explore the information in the time series data, we need to explore the possibility of time series analysis.
Forecasting future values, such as future trends, in time series data is not a new concept, and a lot of work has been done in this area.
Statistical methods such as ARIMA have long been the de facto state-of-the-art method for time series forecasting.
\todo[inline]{Add a reference to ARIMA background and Theory? Add a reference to paper?}
In newer research deep-learning has been explored more in-depth in order to make time-series predictions.
Methods such as Convolutional neural networks \ref{section:BT:CNN} and Long-Short Term Memory \ref{section:BT:LSTM}.
These methods have shown great promise in time series forecasting du to their ability to extract dimensional and temporal features respectively.
In this paper, we wish to explore new and modern approaches to time series forecasting in order to improve the predictive ability in the E-commerce problem space.
By applying new methods to this well-known problem space, we wish to improve upon the current state of the art and contribute to increasing the predictive accuracy of forecasting.

% New section --
% In order to explore the data in the time series, we wish to predict future values in the time series
% Time series forecasting
% Previous state of the art ARIMA (link to Arima from Background and Theory)
% Current solutions are based on deep learning. Wish to explore new and more accurate versions of DL time series preditors
% Thus, we need to do analysis of the data and the current literature
% We end up looking into improvements in the field of time series predictive forecating


