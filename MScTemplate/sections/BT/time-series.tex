\subsection{Time Series}
\label{sec:time-series}
\begin{quote}
    A time series is a sequence of data
    points that occur in successive order over some period of time.
\end{quote}
\cite{Hayes}

In a time series, time is often the independent variable.
Examples of time series are weather data, stock markets, sound level samples.
The time $t$ usually ranges over a discrete index set and is often equally spaced.

\subsubsection{Properties}
A time series has several properties:

\textbf{Stationarity}
A time series is stationary if its statistical properties do not change over time.
In other words, if it has a variance, mean, and covariance which is independent of time.

\cite{RobJHyndman2014} defines stationarity more formally:
\begin{definition}
   $X_t$ is a stationary time series 
   $x_1, ..., x_n, if \forall_s \in \mathbb{R} :$
   the distribution of $(x_t, ..., x_{t+s})$ is equal
\end{definition}

\textbf{Seasonality}
If the time series follows periodic fluctuations, like how electricity usage varies during 24 hours,
then it has seasonality.

\textbf{Autocorrelation}
If a time series has a strong autocorrelation then there is a big
correlation between observations with a time lag between them.

\textbf{Trends}
When a time series has a deterministic component that is proportionate to the time period it has a trend.
In simpler terms, if a time series plot seems to center around an increasing or decreasing line it suggests the presence of a trend.

\textbf{Cycles}
Cycles differ from seasonality because the period does not have to be fixed.


\textbf{Level}
The level of a time series is equal to the mean. If a time series has a trend
then the level is changing.


\subsection{Forecasting time series}
Let  $Y = \{y_1, y_2, ..., y_n\}$ denote a time series.
Forecasting is prediction the next time step $y_{n+h}$ where $h$ is the forecasting horizon.
% Overskriften sier "modeling of time series", men det første du gjør er å definere forecasting?
% Skal det være y_(n+h)? Og burde det ikke isåfall være en serie, ikke en variabel? Mulig jeg misforstår.

There are two main categories within time series forecsasting. \textbf{univariate} and \textbf{multivariate}. 
An \textit{univariate} time series consists of one input variable and one output variable. These methods use the time series past to predict its future.
In a multivariate time series, there are many time dependent variables used as explanatory variables that all help predict one output variable.


Many time series methods focus on predicting just one step ahead ($y_n+1$). When forecasting many steps into the future this becomes a 
\textit{Multi-step forecasting} problem. One simple to predict many steps ahead is to reccursively predict one step ahead, and use past predicted steps 
in the calculation.
% Sikkert egentlig bare jeg som er litt uenig, men kanskje legge til en settning ekstra her for at det blir mer naturlig?


Given a stationary time series, a naive approach to time series modeling is predicting that the next observation will be the
mean of all past observations. A better approach is to define a smaller window, and
apply the moving average across the whole series. Longer window size equals a smoother graph.

A different well-known technique is \textbf{exponential smoothing}. It uses the same approach,
but assigns a different decreasing weight is assigned to each observation.

\begin{equation}
    \label{eq:exponential_smoothing}
    y = \alpha x_t + (1 - \alpha)y_{y-1}, t > 0
\end{equation}

\autoref{eq:exponential_smoothing}
shows exponential smoothing, where $\alpha$ smoothing factor
that takes values between 0 and 1. It determines how fast the weight decreases with time.

\todo{
 TODO:
 Det er jo ikke feil! Men virker som det er veldig mye crammet inn på veldig liten plass.
 Vi har ikke dårlig plass så du har råd til å bruke et par ord ekstra hvis du vil ;)
}

% TODO: Fortsette https://towardsdatascience.com/the-complete-guide-to-time-series-analysis-and-forecasting-70d476bfe775



