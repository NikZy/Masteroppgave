\subsection{Time Series}
\label{sec:time-series}
\begin{quote}
    A time series is a sequence of data
    points that occur in successive order over some period of time.
\end{quote}
\cite{Hayes}

In a time series, time is often the independent variable.
A time series has several properties worth mentioning.
Is it stationary?
A stationary time series if its statistical properties do not change over time.
In other words, if it has a variance, mean and covariance which is independent of time.

Does the time series have seasonality?
If it follows periodic fluctuations, like how electricity usage varies during a 24-hour period,
then it has seasonality.

If a time series has a strong autocorrelation then there is a big
correlation between observations with a time lag between them.

\subsection{Modeling time series}
A naive approach to time series modeling is predicting that the next observation will be the
mean of all past observations. A better approach is to is to define a smaller window, and
apply the moving average across the whole series. Longer window size equals a smoother graph.

Exponential smoothing uses the same approach,
but assigns a different decreasing weight is assigned to each observation.

\begin{equation}
    \label{eq:exponential_smoothing}
    y = \alpha x_t + (1 - \alpha)y_{y-1}, t > 0
\end{equation}

\autoref{eq:exponential_smoothing}
shows exponential smoothing, where $\alpha$ smoothing factor
that takes values between 0 and 1. It determines how fast the weight decreases with time.

% TODO: Fortsette https://towardsdatascience.com/the-complete-guide-to-time-series-analysis-and-forecasting-70d476bfe775


\subsubsection{SARIMA}
SARIMA is an extension to an Autoregressive Integrated Moving Average (ARIMA) model that supports the direct modeling of a seasonal component.

SARIMA is an combination of simpler models to make a complex model that can model time series.
The main idea is to apply different transformations to a non stationary seasonal time series,
in order to remove seasonality and any non-stationary behaviors.
\citet[p. 327-385]{Utlaut2008}.
The first part of SARIMA is the autoregression model
$AR(p)$ where $p$ is the maximum lag.

The second part is the moving average model $MA(q)$ where $q$ is the maximum lag.

The third part is the order of integraion $I(d)$ where $d$ is the number of
differences required to make the series stationary.

The final component is seasonality $S(P, D, Q, s)$, where $s$ is the length
of the season.
$s$ is dependent on $P$ and $Q$ which are equal to $q$ and $q$ but for the seasonal component.
$D$ is the number of differences required to remove seasonality from the series.

The combination of all these parts is the SARIMA model $SARIMA(p, d, q)(P, D, Q, s)$






