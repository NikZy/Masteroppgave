\subsection{Time Series}
\label{sec:time-series}
\begin{quote}
    A time series is a sequence of data
    points that occur in successive order over some period of time.
\end{quote}
\cite{Hayes}

In a time series, time is often the independent variable.
Examples of time series are weather data, stock markets, sound level samples.
The time $t$ usually ranges over a discrete index set and is often equally spaced.

\subsubsection{Properties}
A time series has several properties:

\todo[inline]{Legge inn illustrasjoner som forklarer bedre}

\textbf{Stationarity}
A time series is stationary if its statistical properties do not change over time.
In other words, if it has a variance, mean, and covariance independent of time.

\cite{RobJHyndman2014} defines stationarity more formally:
\begin{definition}
   $X_t$ is a stationary time series 
   $x_1, ..., x_n, if \forall_s \in \mathbb{R} :$
   the distribution of $(x_t, ..., x_{t+s})$ is equal
\end{definition}

\textbf{Seasonality}
If the time series follows periodic fluctuations, like how electricity usage varies during 24 hours,
then it has seasonality.

\textbf{Autocorrelation}
If a time series has a strong autocorrelation then there is a big
correlation between observations with a time lag between them.

\textbf{Trends}
When a time series has a deterministic component that is proportionate to the time period it has a trend.
In simpler terms, if a time series plot seems to center around an increasing or decreasing line it suggests the presence of a trend.

\textbf{Cycles}
Cycles differ from seasonality because the period does not have to be fixed.


\textbf{Level}
The level of a time series is equal to the mean. If a time series has a trend
then the level is changing.


