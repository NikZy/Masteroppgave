
\subsubsection{ARMA}

Auto-Regressive Moving Average \textbf{ARMA} is a statistical model for time-series prediction.
It is one of the most commonly used methods for univariate time-series forecasting.
\textit{ARMA}$(p, q)$ is defined for stationary data and consists of two components $AR(p)$ and $MA(q)$.

The $AR(p)$ model is built on the assumption that the value of a given time-series $y_n$ can be estimated using a linear combination
of the $p$ past observations, an error term $\epsilon_n$ and a constant term $c$ as seen in \Cref{eq:arma_ar(p)} \citep{Box2016}.

\begin{equation}
  \label{eq:arma_ar(p)}
  y_n = c + \sum_{i=1}^{p} \phi_i y_{n-1} + \epsilon_n
\end{equation}
where $\phi_i, \forall i \in \{1, ..., p\} $ denote the model parameters, and $p$ is the order of the model.

The second part $MA(q)$ uses the past errors in a similar fasion \Cref{eq_arma_ma(q)}.
\begin{equation}
  \label{eq_arma_ma(q)}
  y_n = \mu + \sum_{i=1}^{q} \theta_i \epsilon_{n-1} + \epsilon_n
\end{equation}
Here $\mu$ represents the mean of observations. $q$ is the order of the model. $\theta_i, \forall i \in \{1, ..., q\}$ represents the parameters of the model.

Combining the past observations \Cref{eq:arma_ar(p)} and past error terms \Cref{eq_arma_ma(q)} we get the \textit{ARMA}$(p,q)$ model in \Cref{eq:arma}.

\begin{equation}
  \label{eq:arma}
  y_n = c + \sum_{i=1}^{p} \phi_i y_{n-1} + \epsilon_n + \mu + \sum_{i=1}^{q} \theta_i \epsilon_{n-1} + \epsilon_n
\end{equation}



\subsubsection{SARIMA}
SARIMA is an extension to the ARMA model that supports the direct modeling of a seasonal component and incorporates a parameter $d$
to transform a non-stationary time-series into a stationary one.

The main idea is to apply different transformations to a non-stationary seasonal time-series
in order to remove seasonality and any non-stationary behaviors
\citep[p. 327-385]{Utlaut2008}.
The first part of SARIMA is the autoregression model
$AR(p)$ where $p$ is the maximum lag.

The second part is the moving average model $MA(q)$ where $q$ is the maximum lag.

The third part is the order of integration $I(d)$ where $d$ is the number of
differences required to make the series stationary.

The final component is seasonality $S(P, D, Q, s)$, where $s$ is the length
of the season.
$s$ is dependent on $P$ and $Q$, which are equal to $p$ and $q$ but for the seasonal component.
$D$ is the number of differences required to remove seasonality from the series.
% Is it correct that it is equal to q and q? Should both be q?

The combination of all these parts is the SARIMA model,
$SARIMA(p, d, q)(P, D, Q, s)$.

\textbf{ARIMA} is the same method as SARIMA, but without the seasonal component.
