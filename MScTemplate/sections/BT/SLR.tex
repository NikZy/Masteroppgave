
\section{Structured Literature Review Protocol}
\todo[inline]{Skrive alt i presens!}
\todo[inline]{
Here you need to include your structured review protocol including search engine, search words, research questions  (for search, not the masters research questions), inclusion createrias and evaluation Criterias. 
}

A structural literature review (SLR) was conducted to find related work for this project.
The SLR process is based on \cite{AndersKofod-Petersen2018}.
This section will briefly describe our process, deveations from the process,
and changes made during the process.

\subsection{Step 1: Idendification of research}
In order to retrieve all relevant literature to the topic we defined our
research questions and terms.

We included the most known search engines as our sources:

\textit{
ACM digital library,
IEEE Xplore,
ISI web of knowledge,
ScienceDirect,
CiteSeer,
SpringerLink and,
Google Scholar
}

Our search terms are listen in \autoref{tab:search-terms-table-1} and \autoref{tab:search-terms-table-2}.
We used a Notion database to keep track of all the papers. The whole database is available here: \cite{slrdatabase}.
%[\href{https://northern-leech-f32.notion.site/Academic-Writing-ca43d0467e5c40149343d8a85136290f}{Notion SRL database}].
We used another Notion Database to keep track of our searches, and track where how we found each Research Paper
\cite{searchtermtable}

Our method for adding literatur to the Notion database was:
\begin{enumerate}
    \item Pick a source search engine
    \item Search the search engine applying AND $\wedge$ and OR $\vee$ using our search terms
    \item Add the Search Term and Source combination to the Search Term Database.
    \item Title and abstract inclusion screening. Add all remotely relevant papers to Notion, and link them with the correct Search Term Database.
    \item Continue until relevant papers are far in between
    \item Repeat whole process with a new search engine
\end{enumerate}

\subsubsection{Devations}
We did end up with some deviations to the algorithm above.
Some papers we found by accident when researching the topic. 
They were seemed too important to not add, but they were too general of it to be worth
a change to our search terms without including a big set of non-relevant papers.
All papers found by accident was also added with a search term table row documenting 
how the paper was found.

Some of the best papers we found referenced some really good papers. These were also 
added to Notion.

The term \textit{Anomaly detection} was removed (11.10.2021) as a term after our goals shifted away from anomaly detection and towards prediction.
This change was documented in our \textit{Decision Matrix}, which is publicly available here:
\cite{decisionmatrix}.

\begin{table}[htbp]
    \begin{center}
        \begin{tabular}{|c|c|c|c|}\hline\hline
Name&Group 1&Group 2&Group 3\\ \hline
Term 1&CNN&\sout{Anomaly detection}&Autoencoder\\ \hline
Term 2&Convolution Neural Networks&\sout{Extreme values}&Encoder Decoder networks\\ \hline
Term 3&&\sout{Outliers}&  \\ \hline
        \end{tabular}
        \caption{Search Terms table 1}
        \label{tab:search-terms-table-1}
    \end{center}
\end{table}%

\begin{table}[htbp]
    \begin{center}
        \begin{tabular}{|c|c|c|c|}\hline\hline
Name &Group 4&Group 5&Group 6 \\ \hline
Term 1&Time series&LSTM&E-commerce \\ \hline
Term 2 &&Long short term memory&Sales \\ \hline
Term 3& &&\\ \hline 
        \end{tabular}
        \caption{Search Terms table 2}
    \label{tab:search-terms-table-2}
    \end{center}
\end{table}%

\subsection{Filtering by Title and Abstract}
Working with 6 different sources and using a manual but orderly template for processing papers,
we decided to do the rough filtering while searching for papers in order to avoid adding too many irrelevant
papers to our Notion database, and saving time.

In order to filter out papers we had three inclusion criteria (IC).
\begin{description}
    \item[IC1] The studies main concern is time series forecasting.
    \item[IC2] The studies should not be older than from 2015.
    \item[IC3] The study focus on CNN, autoencoder or LSTM. Or a statistical method.
\end{description}
IC1 is to exclude all papers that are not relevant to our goal, such astime series classification and anomaly detection.
We added IC2 because we had to scope down the amount of papers to read. Also time series prediction 
using neural networks really had a boom after 2015. Some exceptions to this inclusion criteria was made,
when papers we found referenced papers before 2015.
IC3 was added to keep the papers within our defined scope of methods. 12.10.2021 we added "Or a statistical method"
to IC3 \citep{decisionmatrix}.

After this step we had 70 papers listed.

\subsection{Quality Assessment}
\todo[inline]{TODO: Når vi har kommet lenger på SLR}
%QC 1	The study abstract is concise and describes the aim and results of the study.
%QC 2	The publisher of the paper / study is a respectable scientific source.
%QC 3	Is there is a clear statement of the aim of the research?
%QC 4	Is the study is put into context of other studies and research?
%QC 5	Are system or algorithmic design decisions justified?
%QC 6	Is the test data set reproducible?
%QC 7	Is the test data set reproducible?
%QC 8	Is the experimental procedure throughly explained and reproducible?
%QC 9	Is it clearly stated in the study which other algorithms the study’s algorithm(s) have been compared with?
%QC 10	Are the performance metrics used in the study explained and justified?
%QC 11	Are the test results thoroughly analysed?
%QC 12	Does the test evidence support the findings presented?