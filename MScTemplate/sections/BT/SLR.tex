
\section{Structured Literature Review Protocol}
\label{section:BT:SLR}

A structural literature review (SLR) was conducted to find related work for this project.
The SLR process is based on \cite{AndersKofod-Petersen2018}.
This section will briefly describe our process, deviations from the process,
and changes made during the process.

\subsection{Step 1: Idendification of research}
In order to retrieve all relevant literature on the topic at hand, we defined our
Research Questions (RQ) and Research Terms (RT).

We included the best-known search engines as our sources:

\textit{
  ACM digital library,
  IEEE Xplore,
  ISI web of knowledge,
  ScienceDirect,
  CiteSeer,
  SpringerLink and,
  Google Scholar
}

Our search terms are listen in \Cref{tab:search-terms-table-1} and \Cref{tab:search-terms-table-2}.
We used a Notion database to keep track of all the papers. The whole database is available here
%\href{https://northern-leech-f32.notion.site/1a5ea8cd5ce64e0bbbdbbe8dae75e295?v=17c60e6672c94f87a0ba27b01a1a9588}{Notion SLR Database}
\cite{slrdatabase}.
We used another Notion Database to keep track of our searches and track where how we found each Research Paper
\cite{searchtermtable}

Our method for adding literature to the Notion database was:
\begin{enumerate}
  \item Pick a source search engine
  \item Search the search engine applying AND $\wedge$ and OR $\vee$ using our search terms
  \item Add the Search Term and Source combination to the Search Term Database.
  \item Title and abstract inclusion screening. Add all remotely relevant papers to Notion, and link them with the correct Search Term Database.
  \item Continue until relevant papers are far in between
  \item Repeat the whole process with a new search engine
\end{enumerate}

\subsubsection{Devations}
We did end up with some deviations to the algorithm above.
Some papers we found by accident when researching the topic.
They were deemed too important not to add.
Their search terms were too general for us to consider adding to the search terms without including several other non-relevant papers.
All papers found by accident were similarly linked with a search term table row documenting
how the paper was discovered.

Some of the best papers we found through the structured review contained references to other relevant papers,
some of which were of high enough quality to add to our list of sources.
These were also added to Notion.

The term \textit{Anomaly detection} was removed (11.10.2021) as a search term after our goals shifted away from anomaly detection and towards time-series prediction.
This change was documented in our \textit{Decision Matrix}, which is publicly available
\cite{decisionmatrix}.

\begin{table}[h]
  \centering
  \begin{tabular}{|c|c|c|c|}\hline
    Name   & Group 1                     & Group 2                  & Group 3                  \\ \hline
    Term 1 & CNN                         & \sout{Anomaly detection} & Autoencoder              \\ \hline
    Term 2 & Convolution Neural Networks & \sout{Extreme values}    & Encoder Decoder networks \\ \hline
    Term 3 &                             & \sout{Outliers}          &                          \\ \hline
  \end{tabular}
  \caption{Search Terms table 1}
  \label{tab:search-terms-table-1}
\end{table}%

\begin{table}[h]
  \begin{center}
    \begin{tabular}{|c|c|c|c|}\hline
      Name   & Group 4     & Group 5                & Group 6    \\ \hline
      Term 1 & Time-series & LSTM                   & E-commerce \\ \hline
      Term 2 &             & Long short term memory & Sales      \\ \hline
      Term 3 &             &                        &            \\ \hline
    \end{tabular}
    \caption{Search Terms table 2}
    \label{tab:search-terms-table-2}
  \end{center}
\end{table}%

\subsection{Filtering by Title and Abstract}
Working with six different sources and using a manual but an orderly template for processing papers,
we decided to do the rough filtering while searching for papers in order to avoid adding a vast amount of irrelevant
papers to our Notion database, thus saving time.

In order to filter out papers, we had three Inclusion Criteria (IC).
\begin{description}
  \item[IC1] The study's main concern is time-series forecasting.
  \item[IC2] The study's should not be older than from 2015.
  \item[IC3] The study focus on CNN, autoencoder, or LSTM. Or a statistical method.
\end{description}
IC1 excludes papers with a topic not relevant to our goals, such as time-series classification and anomaly detection.
IC2 was then added to reduce the number of relevant papers, thus reducing the scope of the structured literature review.
Through initial research on the topic of time-series prediction, we found few relevant papers from before 2015.
Newer research was usually more relevant due to the more modern approaches applied, and thus this criterion was introduced.
However, a small number of relevant papers published prior to 2015 were found to have relevance, and an exception was made for these papers.
Lastly, IC3 was added to keep the papers within our defined scope of methods. 12.10.2021, we added "Or a statistical method."
to IC3. This is documented in the decision matrix \citep{decisionmatrix}.

Each Inclusion criteria passed gave the paper a maximum of 1 point. For IC3, if the paper just covered one of the
four listed methods, it would get 0.25 points. If a paper scored 0 points on either of the Inclusion Criteria, it would be filtered out.

\subsection{Quality Assessment}
In the final stage of filtering, the goal is to assess the quality of the work.
This was done using a set of quality criteria for further screening.
%Quality Criteria (QC) which is taken from \cite{AndersKofod-Petersen2018}.
\begin{itemize}
  \item QC 1	The study abstract is concise and describes the aim and results of the study.
  \item QC 2	The publisher of the paper/study is a reputable scientific source.
  \item QC 3	Is there a clear statement of the aim of the research?
  \item QC 4	Is the study put into the context of other studies and research?
  \item QC 5	Are system or algorithmic design decisions justified?
  \item QC 6	Is the test data set reproducible?
  \item QC 8	Is the experimental procedure thoroughly explained and reproducible?
  \item QC 9	Is it clearly stated in the study which other algorithms the study's algorithms have been compared with?
  \item QC 10	Are the performance metrics used in the study explained and justified?
  \item QC 11	Are the test results thoroughly analyzed?
  \item QC 12 Does the test evidence support the findings presented?
\end{itemize}

The criteria are used with a scoring system.
Yes (1 point), to some degree (0.5 points), and no (0 points).
The maximum possible sum from the Quality Assessment is 12. Adding the points from Inclusion Criteria
increases the maximum possible sum to 15.

The quality assessment resulted in a score range of [7.33, 15].
All papers below "7" were filtered out before
the Quality Assessment.
The overall list is public
\cite{SLR-cutoff}.
In order to limit the scope of papers and ensure the quality of the selected, a threshold of minimum 10 points
was set for a paper to be included.

The final list of included papers contains 23 papers. A screenshot of where we drew the line for papers to include
is presented i the appendix \autoref{cha:slr-cut-off-line}.