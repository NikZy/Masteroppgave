
\section{Evaluation}
\label{section:Discussion:Evaluation}


% We dont have actual results to evaluate.
% Should we drop this section then?

% We can "prove" that deep learning methods perform better than statistical methods on enough data
% We can support that grouping data could improve performance when there is a correlation
% We can support that hybrid methods is better at such solutions
% We can support then the proposed method should be a good fit because it is effective on methods with high fluctuations


% _____Text____:
%The proposed method framework presented in \Cref{section:Architecture} was designed through theoretical work in regards to current state of the art time-series prediction models.

\todo[inline]{Is this section needed? We have no results, so list this in the model architecture section insted? }


\iffalse
  When evaluating your results, avoid drawing grand conclusions, beyond that which your results can infact support. Further, although you may have designed your experiments to answer certain questions, the results may raise other questions in the eyes of the reader. It is important that you study the graphs/tables to look for unusual features/entries and discuss these aswell as discussing the main findings in the results.
\fi

