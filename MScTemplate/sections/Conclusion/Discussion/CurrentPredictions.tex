\subsection{Current state of time-series predicitions}

% Det er her vi skal oppsummere og diskutere hva som er gjort tidligere av tids sere prediction.
% Her skal vi diskutere hvordan tidlgiere state of the art ARIMA, SARIMA, LSTM har vært brukt i vanlig time-series prediciton
% Og hvordan det har vært brukt i e-commerce
% Hva er grunnen til at ting ikke har vært brukt i e-commerce hvis det er gjort i time-series prediction?
% Hvorfor er det viktig å snakke om dette her?
% Det er her du skal flette sammen de viktigste delene fra teori og related work
% Legge spessiel vekt på "statistical vs deep learning", dette er kanskje det viktigste punktet her, og da se dette i forhold til e-commerce prediciton

Our research has found that the current state-of-the-art time-series forecasting is highly dependent
on the given domain. Statistical methods, such as ARIMA and exponential smoothing
are still relevant today. They are simple and easily explainable, and they do not require
too much expertise.

The statistical methods fall short in volatile domains, which are prone to sudden changes,
or changes affected by external changes. They are also limited by their univariate nature, which fails to
incorporate interdependent relationships in domains with many time-series [\Cref{section:RelatedWork:forecasting-ecommerce}, \Cref{section:RelatedWork:Statistical-NN}]

As we uncover in \Cref{section:RelatedWork} Neural Networks, have shown to exceed in complex data-rich, domains. Recurrent neural networks, such as LSTMs being the most
prominent method of choice [\Cref{section:RelatedWork:forecasting-ecommerce}]. However, LSTMs are often used in conjunction with other methods, such as CNNs or Autoencoders,
resulting in a hybrid model [\Cref{section:RelatedWork:Hybrid}].

E-commerce is a notoriously tricky domain to forecast. The environment consists of a multitude of time-series,
some which correlate, others who don't. And the scale of the different series can be of several magnitudes in difference.
The time-series are often noisy and can consist of several recurrent seasons [\Cref{section:RelatedWork:forecasting-ecommerce}].

One method that has seen some promising results in the E-commerce domain, has been to train a global LSTM model across multiple,
related, but not interdependent, time-series.

%\todo[inline]{Har lite å lage på. Litt språk på første avsnitt jeg ville endret, men annet enn det mangler det bare en setning eller 2 om hvordan vi vet det vi påstår.}
%\todo[inline]{Eks: "As we uncover in section ... deep learning methods have shown to achieve supperior performence over statistical methods when the domain contains enought data. However..." etc. Bare et forslag. }

