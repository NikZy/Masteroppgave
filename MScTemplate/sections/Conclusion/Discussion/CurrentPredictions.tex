\subsection{Current state of time-series predicitions}


Our research has found that the current state-of-the-art time-series forecasting is highly dependent
on the given domain. Statistical methods, such as ARIMA and exponential smoothing,
are still relevant today. They are simple and easily explainable, and they do not require
too much expertise.

The statistical methods fall short in volatile domains, prone to sudden changes or changes affected by external changes.
They are also limited by their univariate nature, which fails to
incorporate interdependent relationships in domains with many time-series [\Cref{section:RelatedWork:forecasting-ecommerce}, \Cref{section:RelatedWork:Statistical-NN}]

As we uncover in \Cref{section:RelatedWork} Neural Networks have shown to exceed in complex data-rich domains. Recurrent neural networks, such as LSTMs being the most
prominent method of choice [\Cref{section:RelatedWork:forecasting-ecommerce}]. However, LSTMs are often used in conjunction with other methods, such as CNNs or Autoencoders,
resulting in a hybrid model [\Cref{section:RelatedWork:Hybrid}].

E-commerce is a notoriously tricky domain to forecast. The environment consists of a multitude of time-series,
some which correlate, others who don't. And the scale of the different series can be of several magnitudes in difference.
The time-series are often noisy and can consist of several recurrent seasons [\Cref{section:RelatedWork:forecasting-ecommerce}].

One method that has seen some promising results in the E-commerce domain has been to train a global LSTM model across multiple,
related, but not interdependent, time-series.
