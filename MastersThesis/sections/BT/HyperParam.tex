\subsection{Hyperparameter selection}
% Things to mention -> Grid search, random search, bayesian search, the other one that auto_arima uses?

Machine learning algorithms are heavily dependent on the selected hyperparameters for individual situations.
There are several different approaches to hyperparameter tuning that are used within the field of machine learning.
Some methods approach the problem with heavily exhaustive search methods,
whilst others optimize the search space by selecting only a subspace of the search range for finding optimal parameters for a model.

\subsubsection{Grid search}
Grids search is an example of a hyperparameter tuning method relying on an exhaustive search of the parameter range.
By trying out all possible combinations of parameters, the model is likely to find a combination that fits the model well for the current problem space.
In order to evaluate the models fitted through the extensive search, some sort of performance metric is required in order to compare parameters.
Such metrics can be measured through the use of methods such as cross-validation or through the use of a separate validation set.

In order to tune models, the range of parameters to be selected must be specified beforehand so that it is clear as to which range of parameters should be tuned.
\cite{Geron2017}


\subsubsection{Randomized search}
In cases where the number of parameter combinations is limited, Grid search is a fitting approach.
However, when the range of parameter combinations is large, resulting in high time consumption in order to fit all the different combinations,
other methods are often more relevant to utilize.
One such method is \textit{Randomized search} or \textit{Random search}.
Randomized search reduces the parameter search space by selecting only a limited number of parameter combinations at random.
In contrast to Grid search where the range of values for each parameter is limited as to not create a too large parameter search space,
random search does not share this problem.
The random search method runs only a selected number of iterations before completing, disregarding the range of the search space.
Thus the computational budget of the parameter search is determined by the number of iterations, not by the size of the parameter search space.
\cite{Geron2017}


\subsubsection{Bayesian tuning}
% TODO: https://arxiv.org/abs/1206.2944

Although random search reduces the search space of grid search, the selection of parameters is only random.
The next step is therefore to consider a more informed search of parameters.
One such approach is to use \textit{Bayesian Optimization} or \textit{Bayesian search}.

Bayesian optimization is an optimization method constructing a probabilistic model in order to evaluate parameter selections.
When a sets of parameters are selected, these are compared and the results are observed.
The observed results are used as a measure of generalization of the model performance with the selected hyperparameters.
The method then attempts to optimize the selection of parameters in order to improve the model.

The Bayesian optimization method thusly reduce the search space of hyperparameters using information available from previous evaluations of the model. 

