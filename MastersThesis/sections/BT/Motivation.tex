\section{Motivation*}
\label{section:BT:Motivation}

This section will cover the underlying motivation for the inclusion of the different methods and theories in the literature review.


\subsection{E-commerce and Sales predictions}
With the proposed goal and research questions, the problem presented in this thesis is not to predict sales in an e-commerce setting.
Despite this, the argument is made that the type of data attained from user interest on products should be within a similar data distribution and thus a similar problem space.
Considering this, it is necessary to include a literature review of current solutions in sales predictions and an e-commerce setting.
Researching the current state-of-the-art methods and theory creates a context of what methods have already been investigated and where improvements can be made. 


\subsection{Deep learning methods}
Through the research of the current state-of-the-art forecasting methods,
it is clear that deep learning frameworks are currently at the forefront of time-series forecasting.
Although statistical methods have long been state-of-the-art, current research suggests that deep learning methods can be used to improve predictive accuracy and reduce predictive error.
Research from \cite{Makridakis2018} comparing statistical methods and deep learning methods for time-series forecasting suggests that deep learning methods are superior if there is enough data to process.
Papers such as \cite{Laptev} suggests that methods such as Autoencers and LSTM are well suited for time-series predictions.
Further research into these and other models was therefore needed in order to assess the current development of time-series prediction frameworks.


\subsection{Hybrid models}
Through initial research in the time-series prediction domain,
we discovered a new framework for making time-series predictions.
The paper from \cite{Zhao2019} covers the use of a deep learning framework using a convolutional autoencoder
coupled with an LSTM in order to achieve more accurate predictions.
The proposed framework showed great results on data with high fluctuation,
which would warrant the investigation of its use on our problem space and data.
This hybrid method showed greater predictive ability than the individual parts could achieve independently.
With this, we made a case to find other hybrid models or connected models in order to verify the predictive superiority of hybrid models.
This led us to other papers such as \cite{Khan2020} and \cite{Bowen2020}.

