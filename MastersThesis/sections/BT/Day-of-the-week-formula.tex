\subsection{Day of the week formula}
\label{section:BT:day-of-the-week-formula}
The book \cite{Hale-Evans2006} has an algorithm which can be used to calculate
the day of the week from a date, for example 14.01.1996.
%\todo[inline]{Move to B&T}
\begin{equation}
  day\_of\_the\_week
  = (YC + MC + CC + DC - LYC) \mod{7}
  \label{eq:day_of_the_week}
\end{equation}

\begin{itemize}
  \item YC = Year code.
  \item MC = Month code.
  \item CC = Century code.
  \item DC = Date number.
  \item LYC = Leap year code.
\end{itemize}

The year code is calculated using
$YC = (YY + (YY)/4) \mod{7}$, where $YY$ is the last
two digits of the year. For the year 1996, $YY = 96$.

The \textit{Month Code} is derived from \Cref{table:month_codes}.
The date number \textit{DC} is just what day of the month it is.

The century code \textit{CC} is 4. The leap year code \textit{LYC} is 1 the
month is january or february of a leap year.

\begin{table}[h]
  \centering
  \caption{Month Code table}
  \label{table:month_codes}
  \begin{tabular}{|l|l|}\hline
    Month     & Code \\ \hline
    January   & 0    \\
    February  & 3    \\
    March     & 3    \\
    April     & 6    \\
    May       & 1    \\
    June      & 4    \\
    July      & 6    \\
    August    & 2    \\
    September & 5    \\
    October   & 0    \\
    November  & 3    \\
    December  & 5    \\
    \hline
  \end{tabular}
\end{table}
