\section*{Abstract}
\label{section:Abstract}

% The E-commerce sector is heavily influenced by trends in user activity and purchase patterns.
% 'Prisguiden' is an online service created to compare product prices amongst retailers easily.
% % With this, the site can record the amount of user activity connected to a product or product category.
% To improve service, knowing product trends could help with resource allocation.
% In order to predict trends before they occur, a time-series prediction system could be utilized.


% Time-series prediction in E-commerce sales has been attempted several times,
% often solved by statistical models or deep learning methods.
% Multiple approaches to time-series prediction are proposed in the related literature,
% resulting in many different models and frameworks.


This thesis explores the use of a convolutional autoencoder and LSTM model for making time-series
forecasts on product category trend data supplied by ``Prisguiden.no''.
The use of the CNN-AE-LSTM model is expanded by applying a local univariate, global univariate, local univariate and global multivariate model.
Results are compared with an LSTM baseline model applying the same model types as the CNN-AE-LSTM model.
Additionally, a SARIMA model is used as a predictive baseline for forecasting.


The experiment results show that for the application of the e-commerce data from ``Prisguiden.no'',
the local multivariate LSTM model is the most accurate.
The use of the CNN-AE-LSTM model was explored in comparison with the LSTM model,
showing the model's dependence on the characteristics defining the dataset on which the model is applied.
The model achieves poor performance on data with low levels of noise in the data.
However, predictions made on data with high levels of noise, the CNN-AE-LSTM model achieves
improvements in predictive performance compared to the LSTM model.


The CNN-AE-LSTM model is not well suited for applications with the use of product category trend data from ``Prisguiden.no'',
while the local multivariate LSTM is the model best suited for such predictions.



% Builds on the research done last semester
% Explores the use of global and multivariate LSTM models for improved predictive performance
% Explores the use of the hybrid model convolutional autoencoder and lstm
% Expands to research by applying global and multivariate CNN-AE-LSTM models
% Attempt to predict trends for Prisguiden




\iffalse
  This study conducts an extensive structured literature review of the current state of times series prediction research, E-commerce forecasting,
  and current state-of-the-art methods.
  Subsequently, a hybrid deep neural network model is proposed to increase the predictive accuracy of E-commerce time-series forecasting.
  Combining a convolutional autoencoder with an LSTM network, we discuss the predictive ability of the proposed model
  and its architecture.
  Further, this thesis highlights different strategies for forecasting clusters of related time-series, and discusses their pros and cons.
  % Combining this model with data preprocessing and grouping, we propose the creation of a global prediction framework,
  % grouping related time-series together.
\fi