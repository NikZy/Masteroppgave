\section{Research Method}
\label{section:Introduction:research-method}

% Two approaches
Through this thesis, the goal and research questions defined in \autoref{section:Introduction:Goal} is approached in two different ways.

% RQ1 - RQ3 are done through literature and practical
Initially, research question 1 through 3 is addressed through a theoretical analysis of current literature connected to the subject.
Current state-of-the-art methods and frameworks are compared to assess current valid solutions to the problem space.
Additionally, current work done within the e-commerce sector attempting to predict sales and trends is reviewed.
We argue that this problem space is similar to our proposed problem, and we then use this as a benchmark in order to assess new solutions.
The literature review aims at reviewing the current state of time-series prediction in order to find a method to achieve our goal described in RQ1-RQ3.

% RQ4 - RQ5 are done through practical
Secondly, research questions 4 and 5 are assessed through practical experimentation.
The proposed convolutional autoencoder and LSTM are tested on the available dataset in order to make predictions.
Predictions made by the model can then be compared to predictions done by baselines models on the same dataset.
The baselines are created as a result of the literature search conducted in order to find other current state-of-the-art models.
With the baseline models created, the CNN-AE and LSTM model can be evaluated to assert if the model achieves higher predictive accuracy than other models.
In addition to the proposed CNN-AE-LSTM model,
different model structures are tested and compared to the baseline.
% Lastly, evaluating the number of timesteps giving a meaningful prediction is based on the ability of multi-step prediction of previous state-of-the-art models.



\iffalse
  % Old text
  This thesis approaches the goal and research questions through theoretical analysis
  of the problem space.
  Primarily the focus of this thesis is to conduct a review and analysis of current literature.
  Reviewing the current state-of-the-art methods for predictive analysis of time-series,
  as well as new experimental methods and frameworks.
  Additionally, we review the current work done within the E-commerce sector attempting to predict sales and trends.
  We argue that this problem space is similar to our proposed problem, and we thus use this as a benchmark in order to assess new solutions.
  The literature review aims at reviewing the current state of time-series prediction,
  in order to best find a method to achieve our goal described above.
\fi


%%%%%%%%%%%%%%%%%%%%%%%%%%%%%%%%%%%%%%%%%%%%%%%%%%%%%%%%%%%%%%%%
%%%%%%%%%%%%%%%%%%%% Guideline text %%%%%%%%%%%%%%%%%%%%%%%%%%%%
%%%%%%%%%%%%%%%%%%%%%%%%%%%%%%%%%%%%%%%%%%%%%%%%%%%%%%%%%%%%%%%%

\iffalse
  What methodology will you apply to address the goals: theoretic/analytic, model/abstraction or design/experiment?
  This section will describe the research methodology applied and the reason for this choice of research methodology.

\fi
