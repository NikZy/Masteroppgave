
\section{Discussion}
\label{section:Discussion:Discussion}


This section presents the underlying discussion creating the basis for the model framework proposed in \Cref{section:Architecture}.
The discussion concerns the current state of time-series prediction, the motivation behind the method selection, model structure, and the selected error metric.
This section intendeds to answer the research questions proposed in this paper,
as well as the reason behind the framework.

\import{./sections/Conclusion/Discussion}{datasets_characteristics.tex}

\import{./sections/Conclusion/Discussion}{convolutional-autoencoder-lstm.tex}
\import{./sections/Conclusion/Discussion}{modeling-seasonality.tex}
\import{./sections/Conclusion/Discussion}{global-versus-local-models.tex}
\import{./sections/Conclusion/Discussion}{normalization-vs-standardisation.tex}
\import{./sections/Conclusion/Discussion}{poor-results.tex}



%s\import{./sections/Conclusion/Discussion}{CurrentPredictions.tex}
%s\import{./sections/Conclusion/Discussion}{ProposedFramework.tex}
%s\import{./sections/Conclusion/Discussion}{ModelStructure.tex}
% \import{./sections/Conclusion/Discussion}{ErrorMetric.tex}

% TODO Discussion
% Skrive om forskjellene mellom resultatene mellom datasetene og hva det kan indikere
% dataset 1 sterk ukentli seaason
% dataset 2 svak ukentlig season
% Dataset 2 sterkere årlig season og høyere autocorrelation som gir høyere MASE
% Dataset 3 veldig høy autocorrelation litt dårligere ukentli season
% Kan se ut som SARIMA er bedre enn LSTM til å håndtere årlige sesonger
% Men multivariate LSTM er aller best på årlige sesonger.



