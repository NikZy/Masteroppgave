
\section{Future Work}
\label{sections:Discussion:FutureWork}

%%%%%%%%%%%%%%%%%%%%%%%%%%%%%%%%%%%%%%%%%%%%%%%%%%%%
% New Future work
% - Implement in real world application
% - Prisuiden can apply the resarch to data analysis
% - Further attempts done on splitting / decomposing data
% - Decomposing multivariate ...
% - Apply to a higher forecasting window (Predicting months insted of weeks)
% - There migh be changes in optimal approach hear.
% - We only breefly tried a 30 day forecast window
%%%%%%%%%%%%%%%%%%%%%%%%%%%%%%%%%%%%%%%%%%%%%%%%%%%%

% Implement real world
The work done in this thesis investigates the use of the LSTM and convolutional autoencoder and LSTM
on the e-commerce domain of ``Prisguiden.no''.
However, the research done in this thesis is not entirely practically applicable.
This comes from the fact that only a small sub-set of the entirety of the available data has been applied
in the research done.
This was a limitation applied due to the scope of the thesis and time limitations that adhere to the research done in a master's thesis.
Therefore, applying the models explored in this thesis to a real-world application remains a valid future task.

% There exists other frameworks and models. Test others?
Additionally, while the priority of this thesis was to validate and explore the use of a convolutional autoencoder and LSTM
in a time series prediction setting, other models might also fit well with the task of making such predictions.
The Facebook model Prometheus, XGB, and LightGBM,
are all valid models that can also be tested on the dataset to evaluate their usability.


% Decomposition
% As is described in \Cref{section:Discussion}, different forms of data preprocessing were applied and tested in this thesis.
% While some types of decomposition were applied and tested, others were not tested due to limitations of time.
% \textbf{STR decomposition} was only attempted on univariate models, while their multivariate models were not tested.
% As the decomposition of univariate models improved the accuracy of the univariate predictions,
% so might the use of decomposition with multivariate models or other combinations and applications of decomposing the data.
% \todo[inline]{Chech with what Sindre has written regarding decomposition \dots}


% Increase forecasting window
The research conducted in this thesis focuses primarily on forecasting a period of 7 days.
While a 7-day prediction does have value,
there might be some merits to increasing the forecasting window.
By increasing the forecasting window from 1 week to 30 days (a month), or 60 days (2 months),
the use case for these predictions would empower ``Prisguiden.no'' as discussed \Cref{section:Discussion}.
While a 30 day prediction period was briefly tested,
the 7-day prediction has been prioritized in this research.
Increasing the forecasting window is, therefore, a possible point of entry for further research conducted on the
dataset from ``Prisguiden.no''.


% Unclear what datasets are ideal for the autoencoder
% Explore other factors than noise that define the dataset
% We have only explored the use of noise.
% Selection of datasets correlating and non-correlating were defined more for the use of gloal models, not to explore the use of the autoencoder
% What datasets gave poor results, and which caracteristics gave good results?
Although the experiments done in this thesis are done on several unique datasets,
more work can be done with the experimentation of hybrid model used on datasets with different characteristics.
While dataset 1 and 2 focus on corrolation between time series, this was selected primarily for the experimentation with local vs global models.
The additional testing defined in \Cref{section:Results:AdditionalExperimentalPlan} explores the use of the convolutional autoencoder and LSTM
on datasets with higher levels of noise and lower levels of noise.
The same concept can be applied to create datasets with other characteristics.
By doing this, the hybrid model performance can be explored more in-depth using different characteristics of data.
It is expected that the type of data selected for experimentation is highly influential on the performance of the model,
and further experimentation with the model could therefore be warranted.

E-commerce is a domain that is influenced a lot by external factors, such as holidays and seasonal sales such as black friday.
Expanding the number of features from external sources is a domain worth exploring for future work.


%%%%%%%%%%%%%%%%%%%%%%%%%%%%%%%%%%%%%%%%%%%%%%%%%%%%
%%%%%%%%%%%%%%%%%%%%%%% OLD %%%%%%%%%%%%%%%%%%%%%%%%
%%%%%%%%%%%%%%%%%%%%%%%%%%%%%%%%%%%%%%%%%%%%%%%%%%%%

\iffalse
  This thesis aimed to investigate the current landscape and solutions for time series predictions as a whole,
  as well as in an e-commerce setting.
  Initially, the current predictive ability of solutions in an e-commerce setting was investigated.
  It is then compared to the current state-of-the-art methods for prediction on time series data,
  as well as current methods for grouping and analyzing time series data.
  The thesis focuses on a theoretical approach to solving the proposed research questions,
  excluding practical experiments.

  Work remains on conducting practical experiments on the time series data supplied by Prisguiden,
  in order to make time series predictions on user trends.
  This includes, but is not limited to, creating a predictive baseline using statistical models, such as the ARIMA method, a deep learning baseline such as an LSTM,
  as well as implementing the framework proposed in \Cref{section:Architecture}.
  In addition, practical experiments on the use of a local or global method with the available data,
  as well as the use of a univariate or multivariate method.
\fi
