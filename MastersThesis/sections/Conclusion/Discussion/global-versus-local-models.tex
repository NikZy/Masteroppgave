

\subsection{Global versus Local models}
\label{section:Discussion:Discussion:Global-v-local}

Comparing the local univariate against the global univariate model,
the global model has an sMAPE performance increase of $2.4\%$ on dataset 1, $3.2\%$ on dataset 2 and
$4.9\%$ on dataset 3. The sample sizes of each dataset are relatively low, so the results are not statistically significant.
However, when we increase the sample size by comparing the same models across all the datasets, we see the same trend and we get a lower
p-value. This show a promising trend.
Surprisingly, the performance increase does not seem to correlate
with how homogeneous the time series in the dataset are to each other.
Dataset 1, the most homogeneous set of them, has the least performance increase.
%What might explain these results is that the performance increase is closely connected to the amount
%of data available. Dataset 1 consists of the longest time series.
%Dataset 3 has the least amount of data
%\todo[inline]{Fact check these claims, and add the time series lengths to back up!}
These results support \cite{Montero-Manso2021} preposition that global models can give
improve forecasting accuracy, even if the strong assumption that the same process generates the time series.
And that a global algorithm can also express each forecast that a local algorithm can express.

The same performance increase is not to be found in the multivariate models.
The local multivariate models are good at capturing the trend and seasonality of
a given product category. This does not seem to translate well to a global model.
One might expect that global models should be able to learn seasonality across
time series if the series contains the same seasonality.
The dataset that suffers the most from making a multivariate local model global is dataset 3
with a sMAPE perfomance loss of $-37\%$. Dataset 1 and dataset 2 got $-10.38\%$ and $-6.47\%$ loss
respectively.
One explanation for this might be that even though all categories in dataset 3 are popular during the
winter, and peaks around the same months, their seasonality is not enough in sync.
For example, \textit{"Vintersko"} and \textit{"Vinterjakke"} has their most giant peaks around October
when the weather starts becoming cold. While \textit{"Langrennski"} and \textit{"Skisko"}
peaks around January, when the snow starts falling.
This will seem like conflicting information for an NN that can not differentiate between
which category it is looking at, hurting the NNs modeling capability.
However, the value of additional information is not to be overlooked,
because our results show that a multivariate global LSTM will outperform
a univariate global LSTM in all scenarios.
And the global model is a lot more scalable, which can make
it is an attractive model structure choice, even if it is not the most
accurate model structure.

%TODO:
% Write about which dataset the global model performs best on.
%The global models seems to be doing be doing...
%
%Reasons for why global models does not perform better?
%The time series consist of enough data for the local models to generalize
%\cite{Montero-Manso2021}.

% Difference between MASE ans sMAPE
% This is false on the new results.
%The global models seem to have a better 1-day MASE, but a lower sMAPE.
%This is likely because sMAPE is not a symetric metric, as it punishes
%under forcasts higher than over forecasts. Looking at the predictions made by
%the local and global models, it seems that the global models in general tends to
%under-forcast, and the local models has a tendency to over-forecast.

Regarding RQ4.1 [\ref{RQ4}] our empirical results show that training a global univariate model
across multiple time series will simplify the problem, and improve forecasting accuracy, even if the underlying time series
have very different characteristics.
However, this is not true for multivariate models with additional date stamp features.
We can assume that global multivariate models with date features will work if
all the set time series have the same seasonal patterns.
Local multivariate models will give the most accurate forecast, but global multivariate
models scale a lot better as the number of time series to forecast increases.

%and the strong assumption that the series in the
%set come from the same process, does not have to be true.

%\subsection{LSTM trend and seasonality}
%\begin{itemize}
%  \item LSTM and ARIAM seem to have trouble with datasets with yearly seasonalities
%  \item {LSTM will perform significantly better on these datasets if additional
%        a multivariate version is used with additional information about date}
%  \item {If date is not available then detrending the dataset using differencing is a good second alternative}
%\end{itemize}

