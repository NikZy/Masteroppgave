\section{Conclusion}
\label{section:Discussion:Clonclusion}

%%% Conclusion
% - What have we done?
% --> Tested a new dataset for Prisguiden.no
% --> Tested out new variations of the CNN-AE and LSTM model
% --> These models not tested as far as we find. Thus expand on the domain knowledge
% - Results
% --> The CNN-AE and LSTM is better than LSTM on dataset 1, not on dataset 2 and 3.
% --> Better than naive prediction? Better than SARIMA?
% - Goal and Research Questions
% --> Have the research qiestions been answered?
% --> Have we reached the goal?


The goal of this paper is to forecast future trends using a convolutional autoencoder and lstm model
on a dataset supplied by ``Prisguiden.no''.
% Data recieved from ``Prisguiden.no'' have not been used for time series forecasting before,
As the dataset recived from ``Prisguiden.no'' have not been used for time series forecasting before,
there does not exists any previous baselines for the dataset.
Therefor, the first model to be used on the dataset was the well known SARIMA model,
in order to crete a baseline prediction.
We then expanded the baseline with the LSTM model, answering research question \cref{G&R:RQ-LSTM-baseline}
applying the LSTM to the dataset, and comparing the baseline models.
Thereafter, the LSTM model were expanded and improved using variations of global and local, and univariate and multivariate models
to create comparable results for the LSTM. This was done to answer the research question \cref{G&R:RQ-LSTM-baseline}.
\cref{section:Discussion} explores the resulting models and their predictions, discussing the accuracy of the different LSMT models.
It is clear that the LSTM can be imprived using multivariate models and global models over the standard LSTM model.
These models were also meant to serve as a point of comparison with the convolutional autoencoder and lstm,
in order to evaluate if this model served a purpos and achieved better results.

Answering research question \cref{G&R:RQ-CNN-AE-LSTM} is attempted through the creation of convolutional autoencoder and lstm hybrid models.
These models are compared directly with the LSTM models in order to evaluate if the autoencoder improve the models.
The results were warying, as the hybrid model outperformes the LSTM models for every combination in dataset 1.
However, the LSTM models, spessificaly the local multivariate models performe better than the hybird model on dataset 2 and 3.
The meaning of therse results are explored more in depth in \cref{section:Discussion}.

However, answering research question \cref{G&R:RQ-CNN-AE-LSTM}, we are able to conclude that the hybrid model does \dots
\todo[inline]{Conclude with something regarding the research question.}

\todo[inline]{Conclude with a result regarding the goal! What can we say?}




%%%%%%%%%%%%%%%%%%%%%%%%%%%%%%%%%%%%%%%%%%%%%%%%%%%%%%%%%%%%%%%%%
%%%%%%%%%%%%%%%%%%%%%%%%%%%%% OLD %%%%%%%%%%%%%%%%%%%%%%%%%%%%%%%
%%%%%%%%%%%%%%%%%%%%%%%%%%%%%%%%%%%%%%%%%%%%%%%%%%%%%%%%%%%%%%%%%
% Conclusion points:
% - What has been done (Literature review, Model design)
% - Answer the research questions
% - This is a literature study

\iffalse
The goal of this paper is to propose a method for achieving accurate time-series predictions in an e-commerce
problem space using a hybrid CNN-AE and LSTM method.
In order to achieve this goal, relevant research questions were presented.
Conducting a structured literature analysis of the current status of time-series prediction,
this paper aims to answer the proposed research questions.

Research questions 1 through 3 are addressed through a structured literature review.
Conducting a literature review, current time-series prediction methods are uncovered and compared.
Related work is discussed and reviewed in order to find ways to improve current solutions.
A predictive model is proposed in \Cref{section:Architecture} to solve the goal presented.
This Convolutional autoencoder and LSTM hybrid model is proposed as a method for exploring the goal presented.
The design is influenced by the state-of-the-art methods explored to answer the research questions.
With methods and approaches exploring what is missing from current e-commerce prediction methods,
such as the utilization of correlated time-series, the method aims to address the goal by using the research questions.
\fi
