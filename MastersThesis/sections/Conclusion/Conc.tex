\section{Conclusion}
\label{section:Discussion:Conclusion}

%%% Conclusion
% - What have we done?
% --> Tested a new dataset for Prisguiden.no
% --> Tested out new variations of the CNN-AE and LSTM model
% --> These models not tested as far as we find. Thus expand on the domain knowledge
% - Results
% --> The CNN-AE and LSTM is better than LSTM on dataset 1, not on dataset 2 and 3.
% --> Better than naive prediction? Better than SARIMA?
% - Goal and Research Questions
% --> Have the research qiestions been answered?
% --> Have we reached the goal?

% The goal of this thesis is to forecast future trends using a convolutional autoencoder and LSTM model
% on a dataset supplied by ``Prisguiden.no''.
% % Data recieved from ``Prisguiden.no'' have not been used for time series forecasting before,
% As the dataset received from ``Prisguiden.no'' have not been used for time series forecasting before,
% there do not exist any previous baselines for the dataset.
% Therefore, the first model to be used on the dataset was the well known SARIMA model,
% in order to create a baseline prediction.
% We then expanded the baseline with the LSTM model, answering the research question \Cref{G&R:RQ-LSTM-baseline}
% applying the LSTM to the dataset, and comparing the baseline models.
% Thereafter, the LSTM model was expanded and improved using variations of global and local, and univariate and multivariate models
% to create comparable results for the LSTM.
% This was done to answer the research question RQ2 [\ref{G&R:RQ-LSTM-baseline}].
% \Cref{section:Discussion} explores the resulting models and their predictions, discussing the accuracy of the different LSMT models.
% It is clear that the LSTM can be improved using multivariate models and global models over the standard LSTM model.
% These models were also meant to serve as a point of comparison with the convolutional autoencoder and LSTM,
% in order to evaluate if this model served a purpose and achieved better results.

The goal of this thesis is to forecast future trends using a convolutional autoencoder and LSTM model, explore different model structures,
and create some guidelines for forecasting the dataset supplied by ``Prisguiden.no''.
We reach this goal by answering the research questions RQ4, RQ4.1 and RQ5 from \Cref{section:Introduction:research-method}.

Answering RQ4, we can conclude that a univariate LSTM outperforms SARIMA
unless a strong yearly seasonality is present. However, the LSTM short memory can be
compensated for by adding additional features to the LSTM containing temporal information.
If additional temporal information is unavailable then removing seasonalities with data pre-processing techniques such as differencing,
might improve accuracy on highly seasonal datasets, but will hurt accuracy on non-seasonal datasets.

Further, regarding RQ 4.1, we can conclude that both a multivariate LSTM,
and a global LSTM improves forecasting accuracy in every scenario.
However, the combination of the two model structures is not preferred if accuracy is the prime objective.
The local multivariate LSTM, with features that describe the date, is the preferred method for attaining the lowest predictive error.
However, a global multivariate model can be a good alternative if scalability is essential.

Finally, regarding RQ5  our results,
although not statistically significant, with a p-value of $0.147$,
show that a hybrid Convolutional Autoencoder can improve an LSTM on time series
with a lot of noise.
% Vet ikke helt om jeg kan si dette så hardt? 
However, we can confirm that the hybrid model can severely
hurt accuracy if the datasets are not noisy enough.
Thus, the performance increases are not enough to justify using
a hybrid CNN-AE-LSTM over an LSTM on our dataset.

% --- Old ----
% Answering research question \Cref{G&R:RQ-CNN-AE-LSTM} is attempted through the creation of convolutional autoencoder and lstm hybrid models.
% These models are compared directly with the LSTM models in order to evaluate if the autoencoder improves the models.
% The results were varying, as the hybrid model outperforms the LSTM models for every combination in dataset 1.
% However, the LSTM models, specifically the local multivariate models performed better than the hybrid model on datasets 2 and 3.
% The meaning of these results is explored more in depth in \Cref{section:Discussion}.

% However, answering research question \Cref{G&R:RQ-CNN-AE-LSTM}, we are able to conclude that the hybrid model does
% decrease the predictive error metric for predictions done on data with high noise.
% However, it does not improve predictions on data with low noise; here, the LSTM model is better suited.
% Therefore, the available dataset vastly influences the cases where the hybrid convolutional autoencoder and LSTM is viable for use.

% % TODO: This is a placeholder conclusion.
% % Not sure if this is going to stay here, or if it should be updated.
% When it comes to the other variations of the hybrid model, global univariate models,
% local multivariate models and global univariate models were tested with the time series.
% The local multivariate model is shown to perform better than the local univariate model, as discussed in \Cref{section:Discussion:Discussion:CNN-AE-LSTM}.
% However, while the different configurations of convolutional autoencoders and LSTM often outperform the initial local univariate model,
% they generally perform worse than their LSTM counterpart with the same configuration.
% Therefore, we are not able to conclude that these new configurations of the hybrid model perform any better than the LSTM.
% However, as we have discussed throughout the thesis, the results accomplished are highly dependent on the dataset made available by ``Prisguiden.no'',
% and therefore, we suggest that further experimentations should be conducted on these variations to the convolutional autoencoder and LSTM
% before they are discarded as viable options in some future applications.


% Sindre skrive litt konklusjon uten å tenke på hva som er skrevet før.
% Må merges sammen.


%%%%%%%%%%%%%%%%%%%%%%%%%%%%%%%%%%%%%%%%%%%%%%%%%%%%%%%%%%%%%%%%%
%%%%%%%%%%%%%%%%%%%%%%%%%%%%% OLD %%%%%%%%%%%%%%%%%%%%%%%%%%%%%%%
%%%%%%%%%%%%%%%%%%%%%%%%%%%%%%%%%%%%%%%%%%%%%%%%%%%%%%%%%%%%%%%%%
% Conclusion points:
% - What has been done (Literature review, Model design)
% - Answer the research questions
% - This is a literature study

\iffalse
  The goal of this thesis is to propose a method for achieving accurate time-series predictions in an e-commerce
  problem space using a hybrid CNN-AE and LSTM method.
  In order to achieve this goal, relevant research questions were presented.
  Conducting a structured literature analysis of the current status of time-series prediction,
  this thesis aims to answer the proposed research questions.

  Research questions 1 through 3 are addressed through a structured literature review.
  Conducting a literature review, current time-series prediction methods are uncovered and compared.
  Related work is discussed and reviewed in order to find ways to improve current solutions.
  A predictive model is proposed in \Cref{section:Architecture} to solve the goal presented.
  This Convolutional autoencoder and LSTM hybrid model is proposed as a method for exploring the goal presented.
  The design is influenced by the state-of-the-art methods explored to answer the research questions.
  With methods and approaches exploring what is missing from current e-commerce prediction methods,
  such as the utilization of correlated time-series, the method aims to address the goal by using the research questions.
\fi
