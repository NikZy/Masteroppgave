\section{Threats Against Validty}
\label{section:Discussion:Threats}
All processes are subject to compromises. Some of these comprimises might compromise the validity of
the results. This section summarizes all threats we could find that might affect the results or make
the results difficult to reproduce.

\subsection{Small sample sizes}
We had to limit ourselves to a relatively small sample size. For most experiemnts we have
three datasets, each dataset contains 20 samples, 20, samples, and 8 samples.
Summed up that's 48 samples in total. The high variance in the results, combined with a small sample size
makes it difficult to prove our results with a sufficiently high statistical confidense.
We can however show promising trends, and the fact that the same trends in our result hold, while
the p-value decreases, when
we increase the sample size by comparing across datasets is promising.


% \todo[inline]{Skal vi forsvare valgene våre?}
% Reasons for this small sample size are:
% \begin{itemize}
%   \item Difficult to find categories that sufficiently correlate with each other.
%   \item Amount of experiments we had to run.
%   \item
% \end{itemize}
% LSTM models are computunally expensive to train.
% The combination of experiments we had to run...


\subsection{Generelizability the results}
All of the time series past are relevant for the future. ``Prisguiden.no'' logs new data each day,
so when we executed the final experiments on 26.04.2020 we used all the available data up to that point.
Since the domain of E-commerce is volatile and is constantly changing, we cannot prove that our
results hold for any other points in time. For example, can the buying patterns for our chosen categories
change a lot a couple of years from now.
Therefore we try in this report to focus on the underlying characteristics of the time series, rather than
the specific product categories.


\subsection{Fixed Batch size}
As described in \Cref{section:Method:stateful-lstm-and-hidden-states} we had to set our batch size to
a fixed number. This is not something the papers we base our method on are doing. However,
none explains in detail if they are using a stateful LSTM or how they are implementing it.
If we had solved the problem of batch sizes and stateful LSTM, we might have found that we did not
use the optimal batch size for our problem.

\subsection{Reseting LSTM States}
In section \Cref{section:Method:lstm-reset-hidden-states} we describe how we resets our LSTM states
when training a global model. We describe how we can
in some cases reset the states too late. This is because a batch might contain two different time series.
This can hurt our results.

% \subsection{Problems with Pytorch}
% TODO..?
