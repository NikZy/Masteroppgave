% For mer mat om CNN sjekk denne.
% https://dl.acm.org/doi/pdf/10.1145/3453724
\section{Hybrid frameworks for Time-series forecasting*}
\label{section:RelatedWork:Hybrid}

Improvements in time series forecasting have emerged in the last few years with the introduction of deep learning.
Previous state-of-the-art methods, such as the ARIMA model, have been exchanged for deep learning methods using convolution and recurrent ANNs to improve accuracy.
One such improvement strategy has been to introduce hybrid models.
These models are comprised of different models with unique abilities, helping to increase the accuracy of the connected model.


One example of such a hybrid model is the ARIMA-BP model explored in \cite{Bowen2020}.
This thesis explores the predictive abilities of the ARIMA model and a Back-Propagation Neural network on a time series problem of sales forecasting.
The ARIMA and BP models were applied to the problem, with the ARIMA outperforming the BP model.
However, by combining the two models, the hybrid model was able to achieve higher predictive accuracy than either of the two models accomplished individually.
% This hybrid model is heavily based on the state-of-the-art statistical method of ARIMA.

As we have explored in \Cref{section:RelatedWork:Statistical-NN}, the introduction of more advanced neural networks often achieves better predictive results than the old ARIMA model.
We, therefore, argue that it should make sense to look further into other hybrid models, exploring these new and improved models.


%This combination of models is explored in a few different ways.
The hybrid combination of different ANNs is explored by \cite{Khan2020}, which applies a CNN combined with a LSTM-Autoencoder.
%One such hybrid model is the "CNN and LSTM-Autoencoder" method explored in \cite{Khan2020}.
This method is used to explore the predictive ability of the hybrid model on electricity forecasting in residential and commercial buildings.
The findings of the thesis conclude with the same result as proposed above.
The aforementioned method is a hybrid framework based on convolution, LSTM, and Autoencoder.
Convolution is used to extract features from the input data, while the LSTM-Autoencoder extracts temporal dependencies between the sequences.
The proposed method is applied to the electricity forecasting problem alongside other models such as ARMA, SVM, SVR, and others.
The hybrid framework outperformed the other models at prediction using multiple performance metrics such as MSE, MAE, RMSE, and MAPE,
as well as other hybrid deep learning methods such as a CNN-LSTM.


Similar to the framework described above, another framework applying a combination of the same methods was defined in \cite{Zhao2019}.
This model differs through architecture selection, creating a Convolutional Autoencoder instead of a LSTM Autoencoder.
The Convolutional Autoencoder is used to extract and reduce dimensional features before the LSTM extracts the temporal features.
Similar to the previously mentioned method, this "CNN-AE-LSTM" hybrid method is tested against several other current predictive models.
The proposed method outperforms methods such as LSTM, ARIMA, and SVR, achieving a much lower predictive error than the compared methods.


Other attempts to improve predictive forecasting have also shown promising results.
Frameworks such as with the proposed AE-LSTM from \cite{VanHoa2021} have shown the use of an autoencoder to increase the accuracy of the predictions over a basic LSTM method.
This thesis focuses on foreign exchange rate forecasting and shows the increased accuracy achieved using an Autoencoder and LSTM over a basic CNN or LSTM network.
Similarly, \cite{Zhang2020} uses a Gated Dialated Causal Convolutional Encoder-Decoder in network traffic forecasting.
It shows the decrease in predictive error with an advanced Convolutional Autoencoder compared to a more simple deep learning method like a LSTM.

The literature seems to favor a more complex hybrid ANN over a simpler network.


% \todo[inline]{Add discussion of the referenced methods?}
% ---- Removed ---- 
%With this in mind, we should consider more complex models than basic LSTM and CNN networks to increase predictive accuracy.
%Based on this, we propose that a hybrid model would achieve higher predictive accuracy than the individual parts that it is comprised of.
%Despite the LSTM beeing a well suited method for forecsating prioblems, it is clear that it can be improved upon in several different ways.
%We propose that a framework using a Convoltional-Autoencoder and LSTM would achieve better accuracy than either of its parts could alone.

% It is likely that this would also be the case for a hybrid CNN and LSTM model.

\iffalse
  The proposed problem-space has data with high fluctuations and noise.
  In order to increase the predictive abilities of a model, a method well suited for working with data with high noise should be selected.
  A CNN-AE model should be able to solve this problem.
  The CNN is able to extract the spatial features of the data while the AE can filter out the noise and fluctuations in the data.
  By then adding a LSTM network at the end, the model should be able to extract the temoral featrues from the data.
  This hybrid framework should therefore be well suited for the task at hand.
\fi
