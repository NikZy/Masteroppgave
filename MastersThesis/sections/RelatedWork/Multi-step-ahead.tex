\section{Multi-step ahead}
\label{section:RelatedWork:multi-step-ahead}
Intending to forecast future values, there is a question of how far into the feature a meaningful prediction can be given.
As stated in \Cref{section:BT:forecasting-time-series}, there are two ways to achieve this. 
Either through the use of multiple single-step predictions or a multi-step-ahead forecast.

\cite{Hewamalage2021} presents a section regarding multi-step-ahead forecasting.
Citing \cite{BenTaieb2011} works and their findings that using a multip-input multi-output (MIMO) strategy is
advantageous over the recursive single-step-ahead.
This is because the MIMO strategy incorporates the inter-dependencies between each time step rather than forecasting
each time step in isolation.
They also found that the MIMO strategy voids error
accumulation over the prediction time steps.
\cite{Ramos2015} findings also support the MIMO strategy over the recursive
single-step-ahead strategy.

\cite{Hewamalage2021} suggests that the best input-to-output size ratio
is $window\_input\_size = 1.25 * output\_window\_size$.
With this in mind, the use of multi-step-ahead prediction appears to be the most appropriate approach to use.