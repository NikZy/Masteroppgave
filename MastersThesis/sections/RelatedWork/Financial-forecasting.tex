\section{Forecasting E-commerce}
\label{section:RelatedWork:forecasting-ecommerce}

User click rate on consumer e-commerce goods is a narrow domain,
and we are unable to find any research that has yet been done within this particular field.
However, one can argue that user interest in consumer goods follows the same distribution
as online retail sales.
Both domains will have the same seasonality with weekly and yearly periodic fluctuations.
Additionally, the domains contain the same kind of products and categories.
Both domains are dictated by the same external factors, such as media, commercials,
and trends.
Forecasting retail sales should translate well to e-commerce click rate.


\cite{Ramos2015} did a comparative study on exponential smoothing methods
based on state-space models such as ETS and ARIMA.
The aim was to measure the forecasting performance of the two modeling frameworks
when applied to retail sales data of five different categories,
containing univariate time-series regarding sales of women's footwear.
The domain proved difficult due to strong trends and seasonal patterns.

The models forecasting ability were evaluated through several loss functions, such as RMSE, MAE, and MAPE.
Both ETS and ARIMA achieved similar results with both one-step and multi-step forecasting.
Their results show that multi-step forecasts are generally more accurate.
This is argued to be a result of multi-step to incorporate more up-to-date information.
Neither of the models is best suited for all circumstances.
% Each time-series would need to identify an appropriate ARIMA model.
% With differing data, identifying models would require data analysis of all the different time-series.



\todo[inline]{Rød strek: hva mener anders?}
These models have also been shown to be effective in other instances.
\cite{Chu2003} shows that Holt-Winters exponential smoothing and ARIMA perform well when macro-economic
conditions are relatively stable.
However, when economic conditions are volatile, ANNs outperform linear methods.
Despite this, the existence of stable data is not guaranteed.
Often, the data contain nonlinear characteristics, increasing the complexity of the problem.
\cite{Weng2020} writes that ARIMA, SARIMA, and Holt-Winter's
Exponential Smoothing model etc., is inconsistent with the actual changes in the sales
of retail stores.
Results are usually unstable due to their in-applicability
in the processing of nonlinear relationships.
Despite this, statistical methods have the advantage of high interpretability,
whereas more complex machine learning methods are limited by the high complexity and reduced interpretability.



In comparison to \cite{Weng2020}, the paper \cite{Bowen2020} wanted to use machine learning to maintain
the tradeoff between interpretability and consistency with actual
changes in retail sales.
Using a hybrid ARIMA and Back-propagation method, increased forecasting metrics was achieved through
giving up the interpretability of the pure statistical methods.
% The gain in forecasting ability, but loss of interpretability wasthrough the use of other machine learning methods.
The same gains in forecasting ability but loss of interpretability was
found by \cite{Zunic2020}.
They used pure machine learning methods such as the LightGBM, an ensemble learning method based on the XGB model,
and the Prophet model, with good results.

% What results did they get?
\cite{Bowen2020} finds that traditional models such as ARIMA have trouble forecasting E-commerce
due to the fast-moving domain of consumer retail buying patterns.
Additionally, their connection to external factors,
such as holidays and price changes, increases the complexity further.
The machine learning models performed better, with lower predictive error
and good interpretability.
However, the LightGDM model outperformed the Prophet model because of the
models limitation to univariate time-series.


Further experimentation with machine learning methods is done by \cite{Bandara2019},
through the use of deep learning methods.
An accurate E-commerce sales forecasting model is generated using LSTM  Neural Network.
Similar time-series are pooled into clusters where univariate and global models are trained on each cluster.
The aim was to improve E-commerce sales forecasting by exploiting sales correlations and relationships
available in an E-commerce hierarchy.

Data preprocessing is done using a forward-filling strategy to impute missing sales
observations in the dataset.
The data was then normalized to accommodate the high sales volume ranges.
They then used the mean of the sales of a product as the scaling factor.

Two approaches to groupings were explored.
The first approach was based on domain knowledge.
Sales ranking and the percentage of zero sales were used as the primary business metrics
to form groups of products.
Group 1 represents their most popular products (67\% of sales), group 2 represents
their more unpopular products (33\% of sales), and group 3 represents the rest.

The second approach was based on time-series clustering.
K-means clustering was used on a set of time-series features to identify groupings.
The first two features were business-specific features, namely "sales.quantile" and "zero.sales.percentage".
The rest of the features were time-series specific.
%They extracted features with the \textit{tsfeatures} package.
A silhouette analysis was utilized to determine the optimal number of clusters in the K-means algorithm.

% What results did they get?
The results outperformed the state-of-the-art univariate forecasting techniques.
Thereby concluding that E-commerce product hierarchies contain various cross-product demand
patterns and correlations.
Exploiting these relationships are necessary to improve the sales forecasting
accuracy in this domain.

With the effort to advance the predictive capabilities of models in e-commerce,
previous work done within the problem space is highly relevant as a starting point.
This shows what has already been accomplished and what new methods have not yet been attempted.
