\section{Pytorch vs Keras}
\label{section:method:pytorch-vs-keras}
\todo[inline]{Should we ad the actual experiment results here, or is that too much?}
We did some prototype testing when choosing which framework to use for
building the neural networks.
Our two candidates was Keras and Pytorch.
In two different jupyter notebooks
we implemented a simple LSTM network in both frameworks with the exact same
hyperparameters. Then we ran some experiments, changing the hyperparameters,
but always using the same parameters on both models.
Our results found that Keras outperformed Pytorch by roughly 20\% each time.
These perfomance differences was suriprisingly big between the two.
The only reason we could find that explain this is
\begin{enumerate}
  \item We did something wrong while implementing Pytorch LSTM.
  \item There are some default parameters which we did not know about, and Keras
        default parameters was better tuned for this problem than Pytorch.
\end{enumerate}
Explenation one is plausible, as Pytorch requires a lot more understanding
of the inner workings of a method in order to implement it.
LSTMs are also a quite advanced method. Our Pytorch model did in fact learn
from the data. Nothing seemed wrong on the surface, it just did underperform compared to Keras.

The initial parameters were:
\begin{itemize}
  \item input window: 1
  \item output window: 1
  \item optimizer: Adam
  \item learning rate: 0.001
  \item stateful: false
  \item batch size: 39
  \item hidden state: 50
\end{itemize}
Because of this we ended up using Keras in our experiments. And we can not
garantee reproducible results with the Pytorch framework.