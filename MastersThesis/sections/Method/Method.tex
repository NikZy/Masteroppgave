\chapter{Method}
\label{section:Method}

The applied methodology used in this paper is explore in this chapter.
\cref{section:Method:Hardware&Software} start with pressenting the hardware use for running the experiments.
\cref{section:Method:Metrics} then pressents the error metrics used to measure the accuracy of the predicitons accross models.
Running experiments are done through the use of the framework developed. This framework is explored in more detail in \cref{section:method:experiment-framework}.
This is extended through \cref{section:method:pytorch-vs-keras} exploring the selection of the machine learning framework Kerase and Tensorflow for this project.

Data processing steps used to manipulate data before the use with models are pressented in \cref{section:Method:Preprocessing}.

Lastly, the model methods are defined, exploring the methods used with the SARIMA model \cref{section:Method:SARIMA}, the LSTM model \cref{section:Method:LSTM},
and the hybrid model CNN-AE and LSTM \cref{section:Method:CNN-AE-LSTM}.


% TODO: Write into

The experiments that create the basis for this paper is executed 

Method and methodology for the practical experiments conducted 


With the aim of implementation and evaluation of a new predictive method,

This chapter introduces the implemented method and methodology used in the practical research of time series prediction.
Initially, section \Cref{section:Method:Arima} and 
\Cref{section:Method:LSTM} presents the use and tuning of baseline methods needed in order to validate our method.
\dots


\import{./sections/Method/}{Hardware&Software.tex}
\import{./sections/Method/}{Metrics.tex}
\import{./sections/Method/}{Experiment-framework.tex}
\import{./sections/Method/}{pytorch-vs-keras.tex}

\import{./sections/Method/}{DataProcessing.tex}

% \import{./sections/Method/}{issues-with-lstm.tex}

\import{./sections/Method/}{SARIMA.tex}
\import{./sections/Method/}{LSTM-method.tex}
\import{./sections/Method/}{CNNAE-LSTM.tex}
