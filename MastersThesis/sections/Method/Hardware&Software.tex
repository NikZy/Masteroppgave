
\section{Software and Hardware}
\label{section:Method:Hardware&Software}

% Software
The framework and experiments are implemented using python 3.8 [https://www.python.org/].
The SARIMA method is created using the pmdarima [https://pypi.org/project/pmdarima/] library and
statsmodels library [https://www.statsmodels.org/stable/index.html],
supporting both running the experiments and tuning with auto-arima from the pmdarima package.
On the other hand, the machine learning library Keras is used to implement the deep learning methods used such as the LSTM method
  [https://keras.io/].

When loading data into the experiments, data pipelines were created using the genpipes library [https://pypi.org/project/genpipes/].
After this, data manipulation was conducted using the Pandas library, as well as the numpy library [https://numpy.org/].

% Hardware
The hardware available for running experiments consists of two work-stations.
The first work-station consists of a AMD Ryzen 5 5600X processor and 32 GB 32000 MHz memory with Windows 11 and Linux Sub-system for Linux,
while the second work-station consists of an Intel i9-9900K processor and 16 GB 2666 MHz memory, Windows 10 and Linux Sub-system for Linux.
Both of these work-stations were used for tuning and execution of experiments in this project.

All of the code are open and available at [Github \cite{githubSource}].

