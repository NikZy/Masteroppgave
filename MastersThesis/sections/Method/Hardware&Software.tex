
\section{Software and Hardware}
\label{section:Method:Hardware&Software}

\todo[inline]{Add source for python, Keras, Pandas and genpipes, pmdarima}

% Software
The framework and experiments are implemented using python 3.8 [TODO: Add source].
The SARIMA method is created using the pmdarima library and statsmodels library,
supporting both running the experiments and tuning with auto-aria [TODO: Add source for both].
On the other hand, the machine learning library Keras and Tensorflow is used in order to implement the deep learning methods used such as the LSTM method [TODO: Add Keras and TF source].

When loading data into the experiments, data pipelines were created using the genpipes library.
After this, data manipulation was conducted using the Pandas library, as well as the numpy library.


% Hardware
The hardware available for running experiments consists of two work-stations.
The first work-station consists of a AMD Ryzen 5 5600X processor and 32 GB 32000 MHz memory with Windows 11 and Linux Sub-system for Linux,
while the second work-station consists of an Intel i9-9900K processor and 16 GB 2666 MHz memory, Windows 10 and Linux Sub-system for Linux.
Both of these work-stations were used for tuning and execution of experiments in this project.

