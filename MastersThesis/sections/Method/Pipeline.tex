\section{Pipeline}
\label{section:Method:Pipeline}

% TODO: What is the functionality of the pipeline, and why is it created?
% TODO: Add illustration of the project pipeline

With the aim of running several experiments with ease, a experiment pipeline was created in this project.
Running a project is done through the use of the shared pipeline, through the use of different data pipelines, model structure and model implementations, and different configureations. 

\subsection{Config}
The experiment pipeline contains multiple paramteres that can be adjusted and configured to alter the experiments.
Configuration is done through the use of the ``config.yaml'' file defined in the project source code.
This config file is used to configure the different aspects of the projects to be run.
Ranging from the selected dataset and data files to be used, to the selected model and model structure, with model parameters and random seed.

The condig is parsed through the runntime of the experiment, reading the needed information when it becomes relevant.
The file can be found in the source code at ``./config.yaml''.

\subsection{Data Pipeline}




\subsection{Save sources}
