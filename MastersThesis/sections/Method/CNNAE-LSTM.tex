
\subsection{Hybrid method - CNN-AE and LSTM}

% --------------------------------------------
% __ Contents __
% What model is used?
% What dataset is used? (Is there a split?)
% Error metrics used and recorded?
% Tuning method used
% Experiments run after tuning
% Expectations from experiments
% Add which research questions this answers or helps to answer
% --------------------------------------------


% Model
As described in \cref{section:Architecture:Model}, the Convolutional autoencoder and LSTM model
is based on two individual models.
Initially, the Autoencoder is used to manipulate the models input data,
while the LSTM model thereafter uses the altered input data.

The autoencoder used in this hybrid method is a shared model accross the different time-series
available in the dataset. The same model architecture is used for all models.
\cref{section:Method:AE} describes the development of the autoencoder, alongside with the model architecture selection and tuning.
This autoencoder is then used in the conjoined hybrid model.

Secondly, the LSTM model is added to the hybrid model.
The LSTM model is similar to the model described in \cref{section:Method:LSTM},
where it shares the same dataset, tuning, and architectures.
% There are two different LSTM models that are used. The first is the one tuned by the LSTM. The secound is the one tuned by the hybrid model.



% Dataset (This is the same as the LSTM and the other ones)
% Should this be its own section? I do belive so since it is used accross all methods. Perhaps do this before the pipeline?


% Error metrics
% The same error metrics are used for all the models, except the AE.
% Create a unique section for the error metric selection?


% Tuning
% The tuning is the same as with the LSTM.


% Expectations
% It is expected that the results are better, although marginaly.
% This is due to the high stocastic nature of the dataset



% What research questions are answered?

