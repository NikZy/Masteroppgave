\chapter{Results}
\label{section:Results}

This chapter presents the results accomplished through the execution of the experiments defined in chapter \cref{section:Method}.
First, the results of the ARIMA baselines are presented, containing the tuning of the ARIMA models, as well as the resulting method predictions with these configs.
The second baseline is then pressented, with tuning and testing of the LSTM method.
Lastly, the Convolutional autoencoder and LSTM network is tuned and used on the dataset.


\import{./sections/Results/}{./ARIMA.tex}
% TODO: LSTM
% TODO: CNN-AE and LSTM



\iffalse
\section{Experimental Plan}
\label{sec:experimentalPlan}

Trying and failing is a major part of research. However, to have a chance of success you need a plan driving the experimental research, just as you need a plan for your literature search. Further, plans are made to be revised and this revision ensures that any further decisions made are in line with the work already completed.  

The plan should include what experiments or series of experiments are planned and what question the individual or set of experiments aim to answer. Such questions should be connected to your research questions so that in the evaluation of your results you can discuss the results wrt to the research questions.  

\section{Experimental Setup}
\label{sec:experimentalSetup}

The experimental setup should include all data - parameters etc, that would allow a person to repeat your experiments. 

\section{Experimental Results}
\label{sec:experimentalResults}

Results should be clearly displayed and should provide a suitable representation of your results for the points you wish to make. Graphs should be labeled in a legible font and if more than one result is displayed on the same graph then these should be clearly marked.   Please choose carefully rather than presenting every results. Too much information is hard to read and often hides the key information you wish to present. Make use of statistical methods when presenting results, where possible to strengthen the results.  Further, the format of the presentation of results should be chosen based on what issues in the results you wish to highlight. You may wish to present a subset in the experimental section and provide additional results in the appendix.
\fi