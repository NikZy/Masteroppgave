\subsection{ARIMA}
\label{section:Restuls:ARIMA}


\subsubsection{Tuning}
\label{section:Results:ARIMA:Tuning}
Tuning of the ARIMA models are done through the use of excessive grid search of parameters.
ARIMA parameters are found through the search of parameters for an interval for each of the p,d and q values.

\begin{table}[h]
  \centering
  \caption{The interval of values for the ARIMA parmaeters p,d,q used in grid search tuning.}
  \label{table:results:arima:tuning_parameter_interval}
  \begin{tabular}{|c|l|l|}\hline
    Parameter name & Interval start     & Interval end   \\ \hline
    p   & 1         & 8                 \\ \hline
    d   & 1         & 10                \\ \hline
    q   & 1         & 16                \\ \hline
  \end{tabular}
\end{table}

Tuning is conducted on all of the datasets defined in \cref{section:Architecture:Dataset}.
The tuning is done by calculating the error metric of the one-step ahed predictions done by the ARIMA model on the test set.
Multiple error metrics are used, calculating the MASE, SMAPE, MAE, and MSE of the predictions.
Tabel ... shows a small excerpt of the tuning metrics from dataset 1.
% TODO: Add an exerpt of the parameter tuning of dataset 1
% TODO: Add an exerpt of the parameter tuning of dataset 2
% TODO: Add a reference to complete tuning metrics in the appendix


% Reason for failed tuning (numpy error on linux)
% TODO: Add statsmodels reference!
% TODO: Add LAPACK reference!
With the ARIMA model created using the statsmodels python library, errors were encountered during tuning.
There occures an error with the LAPACK library on which Statsmodels are buildt.
These errors occure only in a few instances when ARIMA models with are trained, using porly selected ARIMA parameters.
The error occures due to a floating point error on different machines, causing some machines to throw an error when the ARIMA parameters cause a stationary model to apear non-stationary du to value precision.
A more detailed descussion on the problem can be found here: https://github.com/statsmodels/statsmodels/issues/5459.

\subsubsection{Experiments}

After completion of tuning as described in \cref{section:Results:ARIMA:Tuning},
the parameters yielding the best results are extracted and applied to experiments.

The ARIMA tuning is conducted using several error metrics such as MASE, SMAPE, MSE and MAE.
Of the implemented error metrics, the MASE error metric is used as a refference.
The parameters which for the ARIMA yeald the best MASE metrics are used as the selected experiment parameters.
Experiments are conducted on both dataset 1 and dataset 2, defined in \cref{section:Architecture:DataSelection}.

% TODO: Add a table with dataset, error metric, metric value, and selected parameters (RESULTS)
% TODO: Add images of the dataset predictions
