
\section{Experimental Plan}
\label{sec:experimentalPlan}


% Template text
\iffalse
  Trying and failing is a major part of research.
  However, to have a chance of success you need a plan driving the experimental research,
  just as you need a plan for your literature search.
  Further, plans are made to be revised and this revision ensures that any further decisions made are in line with the work already completed.

  The plan should include what experiments or series of experiments are planned and what question the individual or set of experiments aim to answer.
  Such questions should be connected to your research questions so that in the evaluation of your results you can discuss the results wrt to the research questions.
\fi


% Experiment 0 - SARIMA Baseline
\subsection{Experiment 0 - SARIMA Baseline}
\label{section:results:experimentPlan:Experiment-0}
This experiment will serve as a measurement for the predictive accuracy of other predictive methods.
The SARIMA baseline will be used to compare the other methods applied in later experiments.
This experiment will therefor help with answering RQ \cref{G&R:RQ-LSTM-baseline}.

\begin{description}
  \item[Outline]{
              Tune, train and test a seasonal SARIMA method on dataset 1, dataset 2,
              and dataset 3 using MASE and sMAPE as metrics on a 7 day forecast horizon.
        }
\end{description}

\begin{description}
  \item[Expectations]{
              The SARIMA method is expected to performe better on more seasonaly dependent data,
              while not performing excedingly well in comparison to other models.
        }
\end{description}

These experiments will serve as the baseline metric for comparison with later experiments.




% Experiment 1 - LSTM Baseline
\subsection{Experiment 1 - LSTM Baseline}
\label{section:results:experimentPlan:Experiment-1}
This experiment will serve as a baseline created using a currently state of the art predictive method
for e-commerce prediction as pressented in \cref{section:Architecture:Baselines:LSTM}.
By comparing the results achieved in this experiment with the ones from \cref{section:results:experimentPlan:Experiment-0}.
This experiment serves to answer RQ \cref{G&R:RQ-LSTM-baseline}.

\begin{description}
  \item[Outline]{
              Tune, train and test a local univariate LSTM on dataset 1,
              dataset 2, and dataset 3 using MASE and sMAPE as metrics on a 7 day forecast horizon.
              The results are compared with the results from the SARIMA baseline. \cref{section:results:experimentPlan:Experiment-0}}
\end{description}

\begin{description}
  \item[Expectations]{
              The univariate local LSTM model is expected to outperforme the SARIMA model on every dataset.
        }
\end{description}




% Experiment 2 - LSTM Model Structure
\subsection{Experiment 2 - LSTM Model structures}
\label{section:results:experimentPlan:Experiment-2}
This experiment attempt to improve the predictive ability of the LSTM model,
using multivariate and global methods.
The experiments are designed to create models with the aim of outperforming the baseline
SARIMA and Local univariate LSTM models created in \cref{section:results:experimentPlan:Experiment-1} and \cref{section:results:experimentPlan:Experiment-0}.
This experiment aims to answer RQ \cref{G&R:RQ-LSTM-baseline},
inn through the exploration of the questions outlined below.

\begin{itemize}
  \item {Will additional information, such as day of the week, month and season help an LSTM to make better predictions?}
  \item {Will giving an LSTM model additional data by training an LSTM across multiple datasets improve predictions?}
\end{itemize}

\begin{description}
  \item[Outline]{
              Tune, train and test a local multivariate LSTM, a global univariate LSTM,
              and a global multivariate LSTM on dataset 1,
              dataset 2, and dataset 3 using MASE and sMAPE as metrics on a 7 day forecast.
              Compare the results against the SARIMA and LSTM baseline.
        }
\end{description}

\begin{description}
  \item[Expectations]{
              It is expected that the multivariate models will outperform the univariate models.
              The additional date-encoding of seasonality should help the NN make more accurate predictions.
              It should not impair the model accuracy.

              Following the works of \cite{Montero-Manso2021} described in \Cref{section:RelatedWork:Model-structure:local-vs-global}
              the global models are expected to outperform at worst, equal to the local models, on all the datasets, and at best outperform the local models.

              %  the local models primarily on dataset 2.
              % This is because dataset 2 consists of independent time series.
              % Following the same logic, accuracy on dataset 1 should be impaired due to the heavy dependence between the time series in the dataset.
              % TODO: This does not make sense. Are we sure that this is correct?
              % TODO: Correlation migh not be the same as dependent
        }
\end{description}





% Experiment 3 - CNN AE and LSTM
\subsection{Experiment 3 - Convolutional Autoencoder LSTM}
This experiment focuses on the proposed hybrid Convolutional Autoencoder and LSTM model.
As this method has yet to be tested on a commercial dataset,
the experiment aims to improve accuracy of the correlating LSTM models.
The hybrid model is explored on each of the experiments ran on the LSTM,
including all combinations of Local vs Global methods, and univariate vs multivariate.

This experiment aims to answer RQ \cref{G&R:RQ-CNN-AE-LSTM} through applying the Hybid method
to the experiment datasets.


\begin{description}
  \item[Outline]{
              Tune, train and test a CNN-AE and LSTM models to compare to the LSTM models
              created in \cref{section:results:experimentPlan:Experiment-1} and \cref{section:results:experimentPlan:Experiment-2}.
              Creating a local-univariate model, local-multivariate model, global-univariate model, and global-multivariate model.
              All these models are applied to each of dataset 1, 2 and 3, using MASE and sMAPE metrics on a 7 day forecast.
              The results are compared against the SARIMA and LSTM models defined in previous experiments.
        }
\end{description}

\begin{description}
  \item[Expectations]{
              The Convolutional Autoencoder and LSTM model is expected to performe better than the correlating LSTM model
              due to the high volatility and noise in the available datasets.
              The multivariate models are expected to performe better than the univariate models.
              However, the global models are expected to performe well on dataset 1 due to the highly correlating data,
              while preforming worse on dataset 2 and 3.
        }
\end{description}